\chapter{Requisiti}
%------------------------------------------------
\section{Introduzione}
%------------------------------------------------
Il cliente richiede lo sviluppo di un'applicazione per la gestione automatizzata degli orari e dei turni dei propri dipendenti.\\

\noindent
Il processo produttivo dell'azienda committente è attivo $7$ giorni a settimana. Ciascuna giornata è suddivisa in tre turnazioni:
\begin{itemize}
	\item 00:00 - 08:00
	\item 08:00 - 16:00
	\item 16:00 - 24:00
\end{itemize}
Nel corso del proprio turno lavorativo, ogni dipendente ha diritto a $30$ minuti continuativi di pausa.\\
Al fine di organizzare al meglio le lavorazioni, per ogni attività del processo produttivo, l'azienda definisce settimanalmente il numero di dipendenti necessari contemporaneamente per il suo svolgimento.\\

\noindent
L'applicazione realizzata dovrà permettere al committente di svolgere le seguenti macro operazioni:
\begin{enumerate}
	\item Gestione dei dati dei dipendenti.
	\item Calcolo degli orari mensili e degli stipendi dei dipendenti.
	\item Allocazione dei dipendenti sui turni.
\end{enumerate}
Dei dipendenti si vogliono gestire i principali dati anagrafici (nome, cognome, data di nascita, etc.), un elenco di attività eseguibili dal dipendente, le informazioni riguardanti la paga base oraria ed il numero di ore settimanali contrattuali.\\

\noindent
Nella sede del cliente è presente una timbratrice che, una volta al giorno, memorizza all'interno di un percorso di rete noto un file \verb|.XML| (un esempio del quale è riportato nella sezione \ref{section:EsempioXML} del capitolo corrente) contenente tutte le timbrature effettuate dai dipendenti durante le $24\;h$ precedenti (indicando anche se è una timbratura di tipo "IN" o "OUT"). Ogni dipendente ha, infatti, un badge personale che permette la timbratura ad ogni entrata o uscita dall'area produttiva (compresa la pausa intermedia).\\
Le informazioni raccolte dalla timbratrice tramite file \verb|.XML| dovranno essere utilizzate dall'applicazione per il calcolo degli orari mensili di ciascun dipendente. In particolare è richiesto il calcolo di:
\begin{itemize}
	\item Ore di lavoro ordinario
	\item Ore di lavoro straordinario
	\item Ore di malattia
	\item Ore di ferie/permessi
\end{itemize}
Queste informazioni devono essere utilizzate anche per il calcolo della paga mensile, considerando che le ore di straordinario e le ore notturne (prima delle 08:00 e dopo le 22:00) sono pagate con un incremento percentuale, configurabile dall'utente.\\
Il committente richiede inoltre di avere una tolleranza di $5$ minuti sugli orari (ad esempio un dipendente che deve iniziare alle 08:00, potrà timbrare fino alle 08:05 senza che venga conteggiato alcun permesso).\\

\noindent
Attualmente, il cliente calcola gli orari dei dipendenti in modo manuale, convertendo il file \verb|.XML| in \verb|.CSV| che viene successivamente importato in Microsoft Excel\textsuperscript\textregistered.\\
Il processo di calcolo degli orari mensili dovrà essere eseguito in automatico dall'applicazione, ma dovrà essere anche presente una  sezione in cui l'addetto dell'ufficio amministrazione potrà eseguire una revisione dei calcoli, o una modifica manuale dei dati.\\
Inoltre l'applicazione dovrà consentire la stampa di un report PDF contenente il quadro generale degli orari mensili di ciascun dipendente.\\

\noindent
Per quanto riguarda l'allocazione dei dipendenti sui turni, attualmente, il committente gestisce le turnazioni tramite accordo privato tra i dipendenti, senza alcuna automatizzazione del processo. L'applicazione dovrà, considerando il numero di dipendenti necessari in ogni turno e per ogni attività, fornire un piano settimanale, avendo cura di non superare il numero di ore contrattuali previste per ciascun dipendente.\\
Dovrà essere anche prevista la possibilità di fare un cambio turno: quando un dipendente non può presentarsi al turno che gli è stato assegnato, deve essere ricalcolato il piano settimanale, in modo da coprire lo slot lasciato scoperto.\\

\noindent
Infine, avendo a disposizione i dati delle timbrature dei dipendenti e la programmazione dei turni, è richiesta l'implementazione di un sistema di controllo, utilizzabile al termine del mese, per verificare l'effettiva corrispondenza tra turni assegnati a ciascun dipendente e relativa presenza in azienda.
%------------------------------------------------
\section{Esempio di file .XML} \label{section:EsempioXML}
%------------------------------------------------
\lstset{
    language=xml,
    tabsize=3,
    frame=lines,
    caption=Esempio di file .XML generato dalla timbratrice,
    label=code:sample,
    frame=shadowbox,
    rulesepcolor=\color{gray},
    xleftmargin=20pt,
    framexleftmargin=15pt,
    morekeywords={xmlns,version,type,encoding},
    keywordstyle=\color{blue}\bf,
    commentstyle=\color{Green},
    stringstyle=\color{red},
    numbers=left,
    numberstyle=\tiny,
    numbersep=5pt,
    breaklines=true,
    showstringspaces=false,
    basicstyle=\footnotesize,
    emph={records,record,badgeID,time},emphstyle={\color{magenta}}}
    
\lstinputlisting[language=XML]{Requisiti/src/Records.xml}
%------------------------------------------------
\section{Tool chain}
%------------------------------------------------
Per la realizzazione del presente software verranno utilizzati i seguenti strumenti:
\begin{itemize}
	\item \underline{Modellazione}
		\begin{enumerate}
			\item \textbf{Use-case diagram, deployment diagram, component diagram}: \textit{UMLet} (\href{http://www.umlet.com}{http://www.umlet.com}), strumento Java open source, molto intuitivo ed utile per la realizzazione di diagrammi (principalmente UML).
			\item \textbf{Class diagram, sequence diagram}: \textit{ObjectAid UML Explorer}, plugin di Eclipse in grado di generare in automatico i diagrammi.
			\item \textbf{Database diagram}: \textit{Vertabelo} (\href{http://www.vertabelo.com}{http://www.vertabelo.com}), strumento online utile per la modellazione dei diagrammi di database e per la generazione automatica dei comandi DDL in codice SQL.
		\end{enumerate}
	\item \underline{Implementazione software}
		\begin{enumerate}
			\item \textbf{DBMS}: \textit{MySQL}, distribuito insieme all'applicativo Apache XAMPP (\href{https://www.apachefriends.org/it/index.html}{https://www.apachefriends.org})
			\item \textbf{Linguaggio di programmazione e IDE}: \textit{JAVA}, tramite l'IDE \textit{Eclipse} (\href{http://www.eclipse.org}{http://www.eclipse.org}).
			\item \textbf{Interfaccia grafica}: \textit{Windowbuilder}, plugin Eclipse che permette la progettazione di interfacce grafiche in maniera drag and drop (\href{https://eclipse.org/windowbuilder/download.php}{https://eclipse.org/windowbuilder/download.php}).
			\item \textbf{Persistenza}: realizzata tramite \textit{JPA 2.0}.
			\item \textbf{Parsing XML}: \textit{SAXParser}, già presente nella libreria standard di Java (\href{https://docs.oracle.com/javase/7/docs/api/javax/xml/parsers/SAXParser.html}{https://docs.oracle.com/javase/7/docs/api/javax/xml/parsers}).
			\item \textbf{Classi e interfacce JCF}: \textit{List}, \textit{ArrayList} e \textit{Vector}, contenute nel package \verb|java.util| (\href{http://docs.oracle.com/javase/tutorial/collections/index.html}{http://docs.oracle.com/javase/tutorial/collections}). \\
				Sono inoltre stati utilizzati alcune funzioni, contenute nella libreria, per lavorare sulle \verb|Collections|, come la \verb|Colelctions.sort|.
		\end{enumerate}
	\item \underline{Analisi del software}
		\begin{enumerate}
			\item \textbf{Analisi statica}: \textit{Stan4J}, plugin di Eclipse (\href{http://stan4j.com}{http://stan4j.com}).
			\item \textbf{Analisi dinamica}: \textit{JUnit}, plugin integrato di default in Eclipse (\href{http://stackoverflow.com/questions/1962567/junit-eclipse-plugin}{http://stackoverflow.com/questions/1962567/junit-eclipse-plugin}).
		\end{enumerate}
	\item \underline{Documentazione e varie}
		\begin{enumerate}
			\item \textbf{Versioning}: \textit{GitHub}, gestito grazie alla relativa applicazione desktop (\href{https://github.com}{https://github.com}).
			\item \textbf{Documentazione}: \textit{LaTex}, con scrittura tramite l'editor \textit{Texpad} (\href{https://www.texpad.com}{https://www.texpad.com})
		\end{enumerate}
\end{itemize}