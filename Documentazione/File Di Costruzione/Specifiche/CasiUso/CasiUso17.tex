%------------------------------------------------
\subsection{UC17: Applicazione incremento percentuale straordinari}
%------------------------------------------------
\paragraph{Descrizione}
%------------------------------------------------
Si vuole applicare un incremento percentuale in caso di lavoro straordinario o di lavoro notturno (prima delle 08:00 e dopo le 22:00).
%------------------------------------------------
\paragraph{Requisiti coperti}
%------------------------------------------------
SM11
%------------------------------------------------
\paragraph{Attori coinvolti}
%------------------------------------------------
Gestione degli orari dei dipendenti
%------------------------------------------------
\paragraph{Precondizioni}
%------------------------------------------------
Ad ogni orario del dipendente è applicata la paga base oraria.
%------------------------------------------------
\paragraph{Postcondizioni}
%------------------------------------------------
Viene applicato l'incremento percentuale sia alle ore di lavoro straordinario che in quelle di lavoro ordinario.
%------------------------------------------------
\paragraph{Processo}
%------------------------------------------------
Di seguito è descritto il processo:
\begin{enumerate}
	\item Il sistema seleziona dal database le ore di lavoro straordinario o notturno.
	\item Il sistema applica l'incremento percentuale alle ore notturne.
	\item Il sistema applica l'incremento percentuale alle ore di straordinario.
	\item I dati vengono memorizzati all'interno del database.
\end{enumerate}

