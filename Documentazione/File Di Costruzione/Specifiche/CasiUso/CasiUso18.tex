%------------------------------------------------
\subsection{UC18: Calcolo stipendio}
%------------------------------------------------
\paragraph{Descrizione}
%------------------------------------------------
Si vuole calcolare lo stipendio mensile di ogni dipendente.
%------------------------------------------------
\paragraph{Requisiti coperti}
%------------------------------------------------
SM12
%------------------------------------------------
\paragraph{Attori coinvolti}
%------------------------------------------------
Impiegato dell'ufficio amministrazione
%------------------------------------------------
\paragraph{Precondizioni}
%------------------------------------------------
Sono caricate le timbrature di ogni dipendente.
%------------------------------------------------
\paragraph{Postcondizioni}
%------------------------------------------------
Per ogni dipendente è calcolato lo stipendio mensile.
%------------------------------------------------
\paragraph{Processo}
%------------------------------------------------
Di seguito è descritto il processo:
\begin{enumerate}
	\item L'impiegato dell'ufficio amministrazione clicca sul pulsante "Calcola stipendi".
		\begin{enumerate}
			\item Viene mostrata una form con la possibilità di selezionare il mese e l'anno di interesse.
		\end{enumerate}
	\item L'impiegato dell'ufficio amministrazione seleziona l'anno ed il mese di interesse.
	\item L'impiegato dell'ufficio amministrazione clicca sul pulsante "Calcola".
		\begin{enumerate}
			\item Il sistema calcola in automatico gli stipendi per tutti i dipendenti.
			\item Nel caso di straordinari o orari notturni, viene applicato un incremento percentuale come da UC17.
			\item Il sistema applica eventuali tassazioni.
			\item Viene mostrata una form con l'indicazione, per ogni dipendente, del relativo stipendio.
		\end{enumerate}
\end{enumerate}
%------------------------------------------------
\paragraph{Alternative}
%------------------------------------------------
\begin{itemize}
	\item \underline{Inserimento anno o mese errato}
		\begin{itemize}
			\item Al passo (2), l'impiegato dell'ufficio amministrazione inserisce dei valori errati per l'anno o il mese. Viene, quindi, mostrato un messaggio di errore e viene data la possibilità di correggere i valori inseriti.
		\end{itemize}
	\item \underline{Inserimento anno o mese futuro}
		\begin{itemize}
			\item Al passo (2), l'impiegato dell'ufficio amministrazione inserisce dei valori futuri per l'anno o il mese (\textit{quindi per i quali ancora non si hanno indicazioni sulle timbrature}). Viene, quindi, mostrato un messaggio di errore e viene data la possibilità di correggere i valori inseriti.
		\end{itemize}
\end{itemize}
