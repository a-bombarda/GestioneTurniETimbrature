%------------------------------------------------
\subsection{UC16: Visualizzazione e calcolo mensile ore di lavoro}
%------------------------------------------------
\paragraph{Descrizione}
%------------------------------------------------
Si vogliono calcolare le ore di lavoro relative ai dipendenti, a partire dai dati derivanti dal parsing del file \verb|.XML| (si veda UC12).
%------------------------------------------------
\paragraph{Requisiti coperti}
%------------------------------------------------
SM05-SM08, SM13, SM24, SM25
%------------------------------------------------
\paragraph{Attori coinvolti}
%------------------------------------------------
Impiegato dell'ufficio amministrazione
%------------------------------------------------
\paragraph{Precondizioni}
%------------------------------------------------
\begin{itemize}
	\item I dati relativi agli orari mensili dei dipendenti non sono presenti.
	\item I dati grezzi (con eventuale modifica manuale o revisione) sono già stati importati dal file \verb|.XML| (si veda UC12).
\end{itemize}
%------------------------------------------------
\paragraph{Postcondizioni}
%------------------------------------------------
Per ogni dipendente sono calcolate le ore mensili di lavoro ordinario, straordinario, di ferie, di permesso e malattia.
%------------------------------------------------
\paragraph{Processo}
%------------------------------------------------
Di seguito è descritto il processo:
\begin{enumerate}
	\item L'impiegato dell'ufficio amministrazione clicca sul pulsante "Visualizza orari dipendenti".
	\item L'impiegato dell'ufficio amministrazione clicca sul pulsante "Visualizza orari mensili".
		\begin{enumerate}
			\item Viene richiesto all'impiegato dell'ufficio amministrazione di inserire il mese e l'anno di interesse.
			\item Viene richiesto all'impiegato dell'ufficio amministrazione di inserire l'identificativo del dipendente per il quale vuole conoscere gli orari mensili.
		\end{enumerate}
	\item L'impiegato dell'ufficio amministrazione inserisce l'anno ed il mese di interesse.
	\item L'impiegato dell'ufficio amministrazione inserisce l'identificativo del dipendente per il quale vuole conoscere gli orari mensili e preme sul pulsante "Conferma" per avviare il processo.
		\begin{enumerate}
			\item Viene avviato il processo di calcolo degli orari del mese selezionato:
				\begin{itemize}
					\item \underline{Lavoro ordinario}: se gli orari rientrano nelle fasce di lavoro ordinario, in base al turno assegnato ogni giorno al dipendente.
					\item \underline{Lavoro straordinario}: se il dipendente supera gli orari di lavoro ordinario.
					\item \underline{Ferie}: se un determinato giorno è stato inserito, per il dipendente considerato, come giorno di ferie (si veda UC14).
					\item \underline{Permesso}: se le ore di ordinario sono inferiori a quelle giornaliere contrattuali.
					\item \underline{Malattia}: se un determinato giorno è stato inserito, per il dipendente considerato, come giorno di malattia (si veda UC14).
				\end{itemize}
			\item Viene mostrata una maschera riassuntiva in cui, per il dipendente desiderato, è indicato il numero di ore mensili di lavoro ordinario, straordinario, di ferie, di permesso e di malattia.
		\end{enumerate}	
\end{enumerate}
%------------------------------------------------
\paragraph{Alternative}
%------------------------------------------------
\begin{itemize}
	\item \underline{Dipendente in ferie presente}
		\begin{itemize}
			\item Al passo (4), il sistema rileva una timbratura per il dipendente in un giorno che è stato precedentemente inserito come giorno di ferie. Viene, quindi, mostrato un messaggio di errore informativo.
		\end{itemize}
	\item \underline{Dipendente in malattia presente}
		\begin{itemize}
			\item Al passo (4), il sistema rileva una timbratura per il dipendente in un giorno che è stato precedentemente inserito come giorno di malattia. Viene, quindi, mostrato un messaggio di errore informativo.
		\end{itemize}
	\item \underline{Superamento limite di lavoro straordinario}
		\begin{itemize}
			\item Al passo (4), il sistema rileva una quantità di ore di lavoro straordinario superiore al limite. In questo caso viene mostrato un messaggio di errore.
		\end{itemize}
\end{itemize}
%------------------------------------------------
\paragraph{Estensioni}
%------------------------------------------------
\begin{itemize}
	\item \underline{Stampa report mensile dipendente}
		\begin{itemize}
			\item Al passo (4), l'impiegato dell'ufficio amministrazione può cliccare sul pulsante "Stampa" per esportare un PDF contenente il report mensile degli orari del dipendente.
			\item Si procede come da UC19.
		\end{itemize}
	\item \underline{Modifica orari timbratura}
		\begin{itemize}
			\item Al passo (3.b), l'impiegato dell'ufficio amministrazione può modificare gli orari di ingresso e uscita del dipendente.
			\item In questo caso, una volta confermata la modifica, viene ripetuto il calcolo delle ore.
		\end{itemize}
\end{itemize}