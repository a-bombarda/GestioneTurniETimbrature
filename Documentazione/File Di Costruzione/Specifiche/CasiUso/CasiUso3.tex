%------------------------------------------------
\subsection{UC3: Modifica dipendente}
%------------------------------------------------
\paragraph{Descrizione}
%------------------------------------------------
Si vogliono modificare alcuni dati di un dipendente.
%------------------------------------------------
\paragraph{Requisiti coperti}
%------------------------------------------------
DD03-DD08, DA01, DA08
%------------------------------------------------
\paragraph{Attori coinvolti}
%------------------------------------------------
Gestore del personale
%------------------------------------------------
\paragraph{Precondizioni}
%------------------------------------------------
Vari utenti sono presenti nel database.
%------------------------------------------------
\paragraph{Postcondizioni}
%------------------------------------------------
Alcuni dati del dipendente scelto sono modificati.
%------------------------------------------------
\paragraph{Processo}
%------------------------------------------------
Di seguito è descritto il processo:
\begin{enumerate}
	\item Il gestore del personale clicca su “Lista dipendenti” nella pagina di gestione dei dipendenti.
		\begin{enumerate}
			\item Viene visualizzata la lista di tutti i dipendenti.
		\end{enumerate}
	\item Il gestore del personale seleziona il dipendente che vuole modificare.	
	\item Il gestore del personale clicca su “Modifica dipendente”. 
		\begin{enumerate}
			\item Viene mostrata la form per la modifica dei dati.
			\item Il gestore del personale modifica i dati.
		\end{enumerate}
	\item Il gestore del personale clicca su “Conferma modifica”.
		\begin{enumerate}
			\item E' richiesta la conferma dell’azione: se l’azione è confermata, il dipendente è modificato.
			\item I nuovi dati del dipendente vengono salvati all’interno del database.
			\item Il dipendente riceve una e-mail contenente la conferma di inserimento ed il riepilogo delle informazioni inserite.
			\item Il sistema mostra un riepilogo indicante, per ogni attività disponibile, quali e quanti dipendenti sono disponibili.
			\item Viene aggiornata la lista di tutti i dipendenti, in modo che si possa eventualmente procedere ad una ulteriore modifica.
		\end{enumerate}
\end{enumerate}
%------------------------------------------------
\paragraph{Alternative}
%------------------------------------------------
\begin{itemize}
	\item \underline{Inserimento dati errati}
		\begin{itemize}
			\item Al passo (3.b) sono inseriti dati errati, in questo caso la modifica è annullata e viene lanciato un messaggio di errore. 
			\item Con “dati errati” si intende nome, cognome, posizione vuoti o formato delle date diverso da gg/mm/aaaa. 
		\end{itemize}
	\item \underline{Dipendente a tempo indeterminato}
		\begin{itemize}
			\item Al passo (3.b) non viene compilato il campo “data fine contratto”. 
			\item Quando si clicca “Conferma modifica” viene lanciato un messaggio di avviso per informare del campo vuoto, ma è comunque possibile proseguire. 
		\end{itemize}
	\item \underline{Attività non presente}
		\begin{itemize}
			\item Al passo (3.b) il gestore del personale non trova l'attività da assegnare al dipendente.
			\item Il gestore del personale clicca sul pulsante “Nuova Attività”.
				\begin{itemize}
					\item Da qua si procede come da UC4.
				\end{itemize}
		\end{itemize}
		\item \underline{Dipendente senza indirizzo e-mail}
		\begin{itemize}
			\item Al passo (3.b) non viene compilato il campo “indirizzo e-mail”. 
			\item Quando si clicca “Prosegui” viene lanciato un messaggio di avviso per informare del campo vuoto, ma è comunque possibile proseguire. 
		\end{itemize}
\end{itemize}