%------------------------------------------------
\subsection{UC13: Gestione straordinari}
%------------------------------------------------
\paragraph{Descrizione}
%------------------------------------------------
Si vogliono inserire/aggiornare o eliminare i dati relativi alle ore straordinarie necessarie in un determinato periodo.
%------------------------------------------------
\paragraph{Requisiti coperti}
%------------------------------------------------
SM17-SM19, SM26
%------------------------------------------------
\paragraph{Attori coinvolti}
%------------------------------------------------
Impiegato dell'ufficio amministrazione
%------------------------------------------------
\paragraph{Precondizioni}
%------------------------------------------------
E' valida una di queste precondizioni:
\begin{itemize}
	\item Non è ancora stato inserito alcun dato riguardante le ore di straordinario necessarie.
	\item I dati sulle ore di straordinario necessarie sono già stati inseriti e devono essere aggiornati.
	\item I dati sulle ore di straordinario necessarie sono già stati inseriti e devono essere eliminati.
\end{itemize}
%------------------------------------------------
\paragraph{Postcondizioni}
%------------------------------------------------
In base all'azione eseguita è valida almeno una delle seguenti postcondizioni:
\begin{itemize}
	\item I nuovi dati riguardanti le ore di straordinario sono inseriti all'interno del database.
	\item I dati riguardanti le ore di straordinario già presenti nel database sono stati aggiornati.
	\item I dati riguardanti le ore di straordinario già presenti nel database sono stati eliminati.
\end{itemize}
%------------------------------------------------
\paragraph{Processo}
%------------------------------------------------
Di seguito è descritto il processo:
\begin{enumerate}
	\item L'impiegato dell'ufficio amministrazione clicca sul pulsante "Gestisci fabbisogno straordinari".
	\item L'impiegato dell'ufficio amministrazione ha a disposizione tre scelte:
		\begin{enumerate}
			\item Può cliccare sul pulsante "Inserisci".
			\item Può cliccare sul pulsante "Modifica".
			\item Può cliccare sul pulsante "Elimina".
		\end{enumerate}	
	\item Se l'impiegato dell'ufficio amministrazione ha cliccato su "Inserisci":
		\begin{enumerate}
			\item L'impiegato dell'ufficio amministrazione inserisce la data per la quale vuole inserire le ore di straordinario necessarie.
			\item L'impiegato dell'ufficio amministrazione inserisce l'attività per la quale sono necessarie le ore di straordinario.
			\item L'impiegato dell'ufficio amministrazione inserisce il numero di ore necessarie.
			\item L'impiegato dell'ufficio amministrazione preme il pulsante "Conferma" per confermare l'inserimento.
				\begin{enumerate}
					\item I dati vengono inseriti nel database.
				\end{enumerate}
		\end{enumerate}
	\item Se l'impiegato dell'ufficio amministrazione ha cliccato su "Modifica":
		\begin{enumerate}
			\item L'impiegato dell'ufficio amministrazione inserisce la data per la quale vuole modificare le ore di straordinario necessarie.
			\item L'impiegato dell'ufficio amministrazione inserisce l'attività per la quale vuole modificare le ore di straordinario necessarie.
				\begin{enumerate}
					\item Il campo contenente le ore di straordinario viene pre-compilato, indicando i dati già inseriti nel database.
				\end{enumerate}
			\item L'impiegato dell'ufficio amministrazione inserisce il nuovo numero di ore necessarie.
			\item L'impiegato dell'ufficio amministrazione preme il pulsante "Conferma" per confermare l'aggiornamento.
				\begin{enumerate}
					\item I dati vengono aggiornati nel database.
				\end{enumerate}
		\end{enumerate}
	\item Se l'impiegato dell'ufficio amministrazione ha cliccato su "Elimina":
		\begin{enumerate}
			\item L'impiegato dell'ufficio amministrazione inserisce la data per la quale vuole eliminare le ore di straordinario necessarie.
			\item L'impiegato dell'ufficio amministrazione inserisce l'attività per la quale vuole eliminare le ore di straordinario necessarie.
			\item L'impiegato dell'ufficio amministrazione preme il pulsante "Conferma eliminazione" per confermare la cancellazione.
				\begin{enumerate}
					\item I dati vengono eliminati dal database.
				\end{enumerate}
		\end{enumerate}	
\end{enumerate}
%------------------------------------------------
\paragraph{Alternative}
%------------------------------------------------
\begin{itemize}
	\item \underline{Inserimento dati errati}
		\begin{itemize}
			\item Ai passi (3.*), (4.*) o (5.*) vengono inseriti dei dati errati. L'operazione viene annullata e viene mostrato un messaggio di errore.
			\item Per dati errati si intende:
				\begin{itemize}
					\item Date non corrette.
					\item Attività non esistenti.
					\item Numero di ore non corretto (\textit{ad esempio $<0$}).
					\item Qualsiasi dato che non rispetta vincoli di dominio (\textit{ad esempio superamento del massimale di ore di straordinario possibili in una giornata}).
				\end{itemize}
		\end{itemize}
	\item \underline{Modifica di una tupla non presente}
		\begin{itemize}
			\item Al passo (4.b) l'impiegato dell'ufficio amministrazione tenta di modificare una tupla non presente. Viene mostrato un messaggio di errore e viene data la possibilità di modificare i dati inseriti.
		\end{itemize}
	\item \underline{Eliminazione di una tupla non presente}
		\begin{itemize}
			\item Al passo (5.c) l'impiegato dell'ufficio amministrazione tenta di eliminare una tupla non presente. Viene mostrato un messaggio di errore e viene data la possibilità di modificare i dati inseriti.
		\end{itemize}
\end{itemize}
