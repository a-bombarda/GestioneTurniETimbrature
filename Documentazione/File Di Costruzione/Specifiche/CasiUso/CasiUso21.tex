%------------------------------------------------
\subsection{UC21: Verifica copertura turni assegnati}
%------------------------------------------------
\paragraph{Descrizione}
%------------------------------------------------
Si vuole verificare che un dipendente abbia coperto tutti i turni che gli sono stati assegnati.
%------------------------------------------------
\paragraph{Requisiti coperti}
%------------------------------------------------
SM25
%------------------------------------------------
\paragraph{Attori coinvolti}
%------------------------------------------------
Impiegato ufficio amministrazione
%------------------------------------------------
\paragraph{Precondizioni}
%------------------------------------------------
Sono già stati caricati i turni per le date che si vogliono analizzare. Inoltre deve già essere caricato anche il dipendente all'interno del database.
%------------------------------------------------
\paragraph{Postcondizioni}
%------------------------------------------------
Viene mostrata una finestra riepilogativa, indicante quali turni sono stati effettivamente coperti dal dipendente, e quali no.
%------------------------------------------------
\paragraph{Processo}
%------------------------------------------------
Di seguito è descritto il processo:
\begin{enumerate}
	\item L'impiegato dell'ufficio amministrazione clicca su "Verifica copertura turni".
		\begin{enumerate}
			\item Viene mostrata una form nella quale l'impiegato dell'ufficio amministrazione può selezionare il dipendente e l'intervallo di interesse.
		\end{enumerate}
	\item L'impiegato dell'ufficio amministrazione inserisce il dipendente per il quale vuole avviare il processo di verifica.
	\item L'impiegato dell'ufficio amministrazione inserisce l'intervallo per il quale vuole avviare il processo di verifica.
	\item L'impiegato dell'ufficio amministrazione clicca su "Verifica".
		\begin{enumerate}
			\item Viene mostrata una form nella quale vengono mostrati i turni che sono stati effettivamente coperti dal dipendente.
		\end{enumerate}
\end{enumerate}
%------------------------------------------------
\paragraph{Alternative}
%------------------------------------------------
\begin{itemize}
	\item \underline{Inserimento date errate}
		\begin{itemize}
			\item Al passo (3), l'impiegato dell'ufficio amministrazione inserisce, per filtrare i risultati, delle date errate. Viene, quindi, mostrato un messaggio di errore e viene data la possibilità di correggere i valori inseriti.
		\end{itemize}
\end{itemize}