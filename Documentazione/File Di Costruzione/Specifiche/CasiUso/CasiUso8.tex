%------------------------------------------------
\subsection{UC8: Generazione turnazione settimanale}
%------------------------------------------------
\paragraph{Descrizione}
%------------------------------------------------
Si vuole generare la turnazione di una nuova settimana, includendo tutti i dipendenti che non sono in ferie.\\
L'assegnazione dei turni deve rispettare il numero massimo di ore settimanali di ciascun dipendente.
%------------------------------------------------
\paragraph{Requisiti coperti}
%------------------------------------------------
TD01, TD02, TD05, TD06, TD07, TD08,  TD12, TD15, DA04-DA08
%------------------------------------------------
\paragraph{Attori coinvolti}
%------------------------------------------------
Impiegato dell'ufficio amministrazione
%------------------------------------------------
\paragraph{Precondizioni}
%------------------------------------------------
\begin{itemize}
	\item I dipendenti sono inseriti all'interno del database del programma.
	\item Non è ancora stata generata una turnazione per la settimana.
\end{itemize}
%------------------------------------------------
\paragraph{Postcondizioni}
%------------------------------------------------
Turni assegnati a tutti i dipendenti per la settimana richiesta, mail (contenente il resoconto della turnazione generata) inviata a ciascun dipendente.
%------------------------------------------------
\paragraph{Processo}
%------------------------------------------------
Di seguito è descritto il processo:
\begin{enumerate}
	\item L'impiegato dell'ufficio amministrazione preme il pulsante "Genera turnazione settimanale".
	\item L'impiegato dell'ufficio amministrazione seleziona la settimana per la quale vuole generare la turnazione.
	\item L'impiegato dell'ufficio amministrazione preme il pulsante "Genera".
		\begin{enumerate}
			\item Il sistema genera in automatico le turnazioni.
			\item Il sistema mostra le turnazioni generate all'impiegato dell'ufficio amministrazione.
		\end{enumerate}
	\item L'impiegato dell'ufficio amministrazione preme il pulsante "Conferma"
		\begin{enumerate}
			\item Le turnazioni assegnate vengono inserite all'interno del database.
			\item Ciascun dipendente riceve una mail contenente le indicazioni sui turni che gli sono stati assegnati per la settimana.
		\end{enumerate}
\end{enumerate}
%------------------------------------------------
\paragraph{Alternative}
%------------------------------------------------
\begin{itemize}
	\item \underline{Selezione settimana errata}
		\begin{itemize}
			\item Al passo (2) l'impiegato dell'ufficio amministrazione seleziona una settimana non esistente o per la quale è già stata assegnata una turnazione. Viene quindi mostrato un messaggio di errore.
		\end{itemize}
	\item \underline{Dipendenti non sufficienti}
		\begin{itemize}
			\item Al passo (3) il sistema non riesce a trovare un'allocazione ammissibile, poichè i dipendenti o la disponibilità di ore non sono sufficienti. Viene quindi mostrato un messaggio di errore.
		\end{itemize}
	\item \underline{Indirizzi e-mail non inseriti}
		\begin{itemize}
			\item Al passo (4) il sistema non trova gli indirizzi e-mail di alcuni dipendenti. Viene mostrata una finestra che comunica i dati mancanti.
		\end{itemize}
\end{itemize}
%------------------------------------------------
\paragraph{Estensioni}
%------------------------------------------------
\begin{itemize}
	\item \underline{Generazione nuova turnazione}
		\begin{itemize}
			\item Al passo (4) l'impiegato dell'ufficio amministrazione clicca su "Genera nuova turnazione". Si riparte quindi dal passo (3.a).
		\end{itemize}
\end{itemize}