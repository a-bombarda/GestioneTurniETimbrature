%------------------------------------------------
\subsection{UC20: Prenotazione per straordinari}
%------------------------------------------------
\paragraph{Descrizione}
%------------------------------------------------
Si vuole fare in modo che un dipendente possa prenotarsi per un certo numero di ore di straordinario previste.
%------------------------------------------------
\paragraph{Requisiti coperti}
%------------------------------------------------
SM28
%------------------------------------------------
\paragraph{Attori coinvolti}
%------------------------------------------------
Dipendente
%------------------------------------------------
\paragraph{Precondizioni}
%------------------------------------------------
Sono già stati caricati i dati relativi alle ore di straordinario necessarie, previste dall'azienda.
%------------------------------------------------
\paragraph{Postcondizioni}
%------------------------------------------------
Nel database è registrata la prenotazione del dipendente per un certo numero di ore di straordinario previste.
%------------------------------------------------
\paragraph{Processo}
%------------------------------------------------
Di seguito è descritto il processo:
\begin{enumerate}
	\item Il dipendente clicca su "Prenotazione straordinari".
		\begin{enumerate}
			\item Viene mostrata una form con l'elenco degli straordinari previsti (si veda UC13), per i quali non sono già terminate le prenotazioni.\\
				Si dà inoltre la possibilità di filtrare le date, inserendo l'intervallo di date desiderato.
		\end{enumerate}
	\item Il dipendente seleziona la riga relativa alla data di interesse.
	\item Il dipendente clicca su "Prenota".
	\item Il dipendente inserisce il numero di ore per cui si prenota.
		\begin{enumerate}
			\item Il dato inserito viene memorizzato nel database.
			\item Viene mostrato al dipendente un messaggio di conferma.
		\end{enumerate}
\end{enumerate}
%------------------------------------------------
\paragraph{Alternative}
%------------------------------------------------
\begin{itemize}
	\item \underline{Inserimento date errate}
		\begin{itemize}
			\item Al passo (2.a), il dipendente inserisce, per filtrare i risultati, delle date errate. Viene, quindi, mostrato un messaggio di errore e viene data la possibilità di correggere i valori inseriti.
		\end{itemize}
	\item \underline{Inserimento ore sopra il limite}
		\begin{itemize}
			\item Al passo (4), il dipendente inserisce un numero di ore superiori a quelle disponibili. Viene, quindi, mostrato un messaggio di errore e viene data la possibilità di correggere i valori inseriti.
			\item Lo stesso comportamento si ha quando il dipendente si prenota, nella stessa settimana o nello stesso mese, per un numero di ore che eccede quelle permesse legalmente.
		\end{itemize}
	\item \underline{Prenotazione per date passate}
		\begin{itemize}
			\item Al passo (3), il dipende cerca di prenotarsi per una data passata. Viene, quindi, mostrato un messaggio di errore.
		\end{itemize}
\end{itemize}
%------------------------------------------------
\paragraph{Estensioni}
%------------------------------------------------
\begin{itemize}
	\item \underline{Selezione date per filtro}
		\begin{itemize}
			\item Al passo (2.a) il dipendente può inserire una o due date, per filtrare quelle memorizzare.
			\item In questo caso vengono mostrate solamente le date con straordinari disponibili, comprese all'interno dell'intervallo.
		\end{itemize}
\end{itemize}