%------------------------------------------------
\subsection{UC19: Generazione report dipendente}
%------------------------------------------------
\paragraph{Descrizione}
%------------------------------------------------
Si vuole stampare un report, per un dipendente selezionato, che rispecchia la situazione mensile lavorativa.
%------------------------------------------------
\paragraph{Requisiti coperti}
%------------------------------------------------
SM03, SM13
%------------------------------------------------
\paragraph{Attori coinvolti}
%------------------------------------------------
Impiegato dell'ufficio amministrazione
%------------------------------------------------
\paragraph{Precondizioni}
%------------------------------------------------
Sono già stati caricati i dati di ore di ordinario, straordinario, ferie, permessi e malattia per il dipendente.
%------------------------------------------------
\paragraph{Postcondizioni}
%------------------------------------------------
Viene prodotto un report mensile del dipendente, con statistiche riguardanti le varie tipologie di lavoro svolto, resoconto delle timbrature e stipendio.
%------------------------------------------------
\paragraph{Processo}
%------------------------------------------------
Di seguito è descritto il processo:
\begin{enumerate}
	\item L'impiegato dell'ufficio amministrazione clicca sul pulsante "Stampa report dipendente".
		\begin{enumerate}
			\item Viene mostrata una form con la possibilità di selezionare il mese e l'anno di interesse, oltre che l'identificativo del dipendente per il quale si vuole ottenere il report.
		\end{enumerate}
	\item L'impiegato dell'ufficio amministrazione seleziona l'anno ed il mese di interesse.
	\item L'impiegato dell'ufficio amministrazione seleziona l'identificativo del dipendendente di interesse.
	\item L'impiegato dell'ufficio amministrazione clicca sul pulsante "Genera".
		\begin{enumerate}
			\item Il sistema aggrega tutte le informazioni necessarie e produce il report PDF per il dipendente selezionato.
		\end{enumerate}
\end{enumerate}
%------------------------------------------------
\paragraph{Alternative}
%------------------------------------------------
\begin{itemize}
	\item \underline{Inserimento anno o mese errato}
		\begin{itemize}
			\item Al passo (2), l'impiegato dell'ufficio amministrazione inserisce dei valori errati per l'anno o il mese. Viene, quindi, mostrato un messaggio di errore e viene data la possibilità di correggere i valori inseriti.
		\end{itemize}
	\item \underline{Inserimento anno o mese futuro}
		\begin{itemize}
			\item Al passo (2), l'impiegato dell'ufficio amministrazione inserisce dei valori futuri per l'anno o il mese (\textit{quindi per i quali ancora non si hanno indicazioni sulle timbrature}). Viene, quindi, mostrato un messaggio di errore e viene data la possibilità di correggere i valori inseriti.
		\end{itemize}
	\item \underline{Inserimento identificativo dipendente errato}
		\begin{itemize}
			\item Al passo (3), l'impiegato dell'ufficio amministrazione inserisce un identificativo di un dipendente non esistente. Viene, quindi, mostrato un messaggio di errore e viene data la possibilità di correggere i valori inseriti.
		\end{itemize}
\end{itemize}
