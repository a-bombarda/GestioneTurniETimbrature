%------------------------------------------------
\subsection{UC7. Allocazione dipendenti per ogni attività}
%------------------------------------------------
\paragraph{Descrizione}
%------------------------------------------------
Si vuole impostare il numero di dipendenti necessari per ogni attività in un intervallo di date.
%------------------------------------------------
\paragraph{Requisiti coperti}
%------------------------------------------------
DA04-DA08
%------------------------------------------------
\paragraph{Attori coinvolti}
%------------------------------------------------
Gestore del personale
%------------------------------------------------
\paragraph{Precondizioni}
%------------------------------------------------
Sono presenti varie attività nel database.
%------------------------------------------------
\paragraph{Postcondizioni}
%------------------------------------------------
Il numero impostato di dipendenti necessari per ogni attività viene inserito all’interno del database.
%------------------------------------------------
\paragraph{Processo}
%------------------------------------------------
Di seguito è descritto il processo:
\begin{enumerate}
	\item Il gestore del personale preme il tasto “Imposta richiesta manodopera per attività”.
	\item Il gestore del personale seleziona l’intervallo di date all’interno del quale vuole impostare la richiesta di manodopera necessaria per attività. 
	\item Il gestore del personale seleziona l'attività.
	\item Il gestore del personale seleziona il turno per il quale vuole impostare la richiesta di manodopera.
	\item Il gestore del personale seleziona il numero di dipendenti, che eseguono l'attività richiesta, necessari per il turno selezionato.
	\item Il gestore del personale clicca sul pulsante “Conferma”.
		\begin{enumerate}
			\item I nuovi dati vengono inseriti all’interno del database.
			\item Il sistema mostra un riepilogo indicante i dati delle attività attualmente presenti nel database, con relative richieste di dipendenti, per l’intervallo di date selezionato.
		\end{enumerate}
\end{enumerate}
%------------------------------------------------
\paragraph{Alternative}
%------------------------------------------------
\begin{itemize}
	\item \underline{Inserimento data errata}
		\begin{itemize}
			\item Al passo (2) il gestore del personale inserisce delle date non valide o con formato non corretto. Viene mostrato un messaggio di errore e viene data la possibilità di modificare le due date.
		\end{itemize}
	\item \underline{Inserimento numero dipendenti errato}
		\begin{itemize}
			\item Al passo (5) il gestore del personale inserisce un valore numerico negativo. Viene mostrato un messaggio di errore e viene data la possibilità di modificare il valore.
		\end{itemize}
\end{itemize}