%------------------------------------------------
\subsection{UC5: Cancellazione di una attività}
%------------------------------------------------
\paragraph{Descrizione}
%------------------------------------------------
Si vuole eliminare una attività precedentemente inserita.
%------------------------------------------------
\paragraph{Requisiti coperti}
%------------------------------------------------
DA02, DA05, DA06
%------------------------------------------------
\paragraph{Attori coinvolti}
%------------------------------------------------
Gestore del personale
%------------------------------------------------
\paragraph{Precondizioni}
%------------------------------------------------
Sono presenti varie attività nel database.
%------------------------------------------------
\paragraph{Postcondizioni}
%------------------------------------------------
L'attività viene cancellata dal database.
%------------------------------------------------
\paragraph{Processo}
%------------------------------------------------
Di seguito è descritto il processo:
\begin{enumerate}
	\item Il gestore del personale preme il tasto “Lista attività”.
		\begin{enumerate}
			\item Viene visualizzata la lista di tutte le attività.
		\end{enumerate}
	\item Il gestore del personale seleziona una attività. 
	\item Il gestore del personale clicca su "Elimina attività".
		\begin{enumerate}
			\item E' richiesta la conferma dell’azione: se l’azione è confermata, l'attività è eliminata.
			\item Viene aggiornata la lista di tutte le attività, in modo che si possa eventualmente procedere ad una ulteriore eliminazione.
		\end{enumerate}
\end{enumerate}
%------------------------------------------------
\paragraph{Alternative}
%------------------------------------------------
\begin{itemize}
	\item \underline{Attività assegnata a dipendenti}
		\begin{itemize}
			\item Al passo (3) si sta cercando di eliminare una attività assegnata ancora da almeno un dipendente, la cancellazione è annullata e viene visualizzato un messaggio di errore.
			\item In questa situazione sarà necessario, in via preventiva, rimuovere l'attività dalla lista di quelle ricoperte dal dipendente, come da UC3.
		\end{itemize}
\end{itemize}
