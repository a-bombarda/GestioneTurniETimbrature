%------------------------------------------------
\subsection{UC6: Modifica di una attività}
%------------------------------------------------
\paragraph{Descrizione}
%------------------------------------------------
Si vuole modificare un'attività precedentemente inserita.
%------------------------------------------------
\paragraph{Requisiti coperti}
%------------------------------------------------
DA03, DA05-DA08
%------------------------------------------------
\paragraph{Attori coinvolti}
%------------------------------------------------
Gestore del personale
%------------------------------------------------
\paragraph{Precondizioni}
%------------------------------------------------
Sono presenti varie attività nel database.
%------------------------------------------------
\paragraph{Postcondizioni}
%------------------------------------------------
L'attività viene modificata nel database.
%------------------------------------------------
\paragraph{Processo}
%------------------------------------------------
Di seguito è descritto il processo:
\begin{enumerate}
	\item Il gestore del personale preme il tasto “Lista attività”.
		\begin{enumerate}
			\item Viene visualizzata la lista di tutte le attività.
		\end{enumerate}
	\item Il gestore del personale seleziona una attività. 
	\item Il gestore del personale clicca su "Modifica attività".
		\begin{enumerate}
			\item Viene mostrata una form nella quale è possibile modificare i vari dati.
			\item Il gestore del personale modifica i dati.
		\end{enumerate}
	\item Il gestore del personale clicca su "Conferma modifica".
		\begin{enumerate}
			\item E' richiesta la conferma dell’azione: se l’azione è confermata, l'attività è modificata.
			\item I nuovi dati dell'attività sono inseriti all’interno del database.
			\item Il sistema mostra un riepilogo indicante i dati delle attività attualmente presenti nel database.
		\end{enumerate}
\end{enumerate}
%------------------------------------------------
\paragraph{Alternative}
%------------------------------------------------
\begin{itemize}
	\item \underline{Inserimento dati errati}
		\begin{itemize}
			\item Al passo (3) sono inseriti dati errati, in questo caso l’inserimento è annullato e viene lanciato un messaggio di errore. 
			\item Con “dati errati” si intende codice, nome, postazione o descrizione vuoti.
		\end{itemize}
\end{itemize}
%------------------------------------------------
\paragraph{Estensioni}
%------------------------------------------------
\begin{itemize}
	\item \underline{Stampa dati di riepilogo}
		\begin{itemize}
			\item Al passo (4.c), il gestore del personale può cliccare sul pulsante “Stampa”, per stampare la scheda di riepilogo contenente i dati delle attività inserite.  
		\end{itemize}
\end{itemize}
