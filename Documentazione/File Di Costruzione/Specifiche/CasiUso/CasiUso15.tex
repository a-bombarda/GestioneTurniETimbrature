%------------------------------------------------
\subsection{UC15: Visualizzazione e calcolo giornaliero ore di lavoro}
%------------------------------------------------
\paragraph{Descrizione}
%------------------------------------------------
Si vogliono calcolare le ore di lavoro relative ai dipendenti, a partire dai dati derivanti dal parsing del file \verb|.XML| (si veda UC12).
%------------------------------------------------
\paragraph{Requisiti coperti}
%------------------------------------------------
SM00-SM02, SM04, SM25, SM27
%------------------------------------------------
\paragraph{Attori coinvolti}
%------------------------------------------------
Impiegato dell'ufficio amministrazione
%------------------------------------------------
\paragraph{Precondizioni}
%------------------------------------------------
\begin{itemize}
	\item I dati relativi agli orari giornalieri dei dipendenti non sono presenti.
	\item I dati grezzi (con eventuale modifica manuale o revisione) sono già stati importati dal file \verb|.XML| (si veda UC12).
\end{itemize}
%------------------------------------------------
\paragraph{Postcondizioni}
%------------------------------------------------
Per ogni dipendente sono calcolate le ore giornaliere di lavoro ordinario, straordinario, di ferie, di permesso e malattia.
%------------------------------------------------
\paragraph{Processo}
%------------------------------------------------
Di seguito è descritto il processo:
\begin{enumerate}
	\item L'impiegato dell'ufficio amministrazione clicca sul pulsante "Visualizza orari dipendenti".
	\item L'impiegato dell'ufficio amministrazione clicca sul pulsante "Visualizza orari giornalieri".
		\begin{enumerate}
			\item Viene richiesto all'impiegato dell'ufficio amministrazione di inserire la data di interesse.
		\end{enumerate}
	\item L'impiegato dell'ufficio amministrazione inserisce la data di interesse e preme sul pulsante "Conferma" per avviare il processo.
		\begin{enumerate}
			\item Viene avviato il processo di calcolo degli orari del giorno selezionato:
				\begin{itemize}
					\item \underline{Lavoro ordinario}: se gli orari rientrano nelle fasce di lavoro ordinario, in base al turno assegnato a ciascun dipendente.
					\item \underline{Lavoro straordinario}: se un dipendente supera gli orari di lavoro ordinario.
					\item \underline{Ferie}: se il giorno d'interesse è stato inserito, per il dipendente considerato, come giorno di ferie (si veda UC14).
					\item \underline{Permesso}: se le ore di ordinario sono inferiori a quelle giornaliere contrattuali.
					\item \underline{Malattia}: se il giorno d'interesse è stato inserito, per il dipendente considerato, come giorno di malattia (si veda UC14).
				\end{itemize}
			\item Viene mostrata una maschera riassuntiva in cui, per ogni dipendente, è indicato il numero di ore giornaliere di lavoro ordinario, straordinario, di ferie, di permesso e di malattia.
		\end{enumerate}	
\end{enumerate}
%------------------------------------------------
\paragraph{Alternative}
%------------------------------------------------
\begin{itemize}
	\item \underline{Dipendente in ferie presente}
		\begin{itemize}
			\item Al passo (3), il sistema rileva una timbratura per un dipendente che è considerato in ferie, in base ai dati inseriti. Viene, quindi, mostrato un messaggio di errore informativo.
		\end{itemize}
	\item \underline{Dipendente in malattia presente}
		\begin{itemize}
			\item Al passo (3), il sistema rileva una timbratura per un dipendente che è considerato in malattia, in base ai dati inseriti. Viene, quindi, mostrato un messaggio di errore informativo.
		\end{itemize}
\end{itemize}
%------------------------------------------------
\paragraph{Estensioni}
%------------------------------------------------
\begin{itemize}
	\item \underline{Modifica orari timbratura}
		\begin{itemize}
			\item Al passo (3.b), l'impiegato dell'ufficio amministrazione può modificare gli orari di ingresso e uscita del dipendente.
			\item In questo caso, una volta confermata la modifica, viene ripetuto il calcolo delle ore.
		\end{itemize}
\end{itemize}