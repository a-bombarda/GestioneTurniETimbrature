%------------------------------------------------
\subsection{UC11: Cambio turno dipendente}
%------------------------------------------------
\paragraph{Descrizione}
%------------------------------------------------
Si vuole, una volta ricevuta una specifica richiesta da un dipendente, modificare il relativo turno. La modifica di questo turno dovrà comportare, eventualmente, la modifica del turno di altri dipendenti.
%------------------------------------------------
\paragraph{Requisiti coperti}
%------------------------------------------------
TD01, TD02, TD09, TD10, TD11, TD12, TD15
%------------------------------------------------
\paragraph{Attori coinvolti}
%------------------------------------------------
Impiegato dell'ufficio amministrazione
%------------------------------------------------
\paragraph{Precondizioni}
%------------------------------------------------
\begin{itemize}
	\item I dipendenti sono inseriti all'interno del database del programma.
	\item E' già stata generata una turnazione per la giornata richiesta.
\end{itemize}
%------------------------------------------------
\paragraph{Postcondizioni}
%------------------------------------------------
\begin{itemize}
	\item Turni modificati, generando un assegnamento consistente.
	\item Mail contenente i nuovi turni inviata ai dipendenti che hanno subìto variazioni.
\end{itemize}
%------------------------------------------------
\paragraph{Processo}
%------------------------------------------------
Di seguito è descritto il processo:
\begin{enumerate}
	\item L'impiegato dell'ufficio amministrazione preme il pulsante "Modifica turno dipendente".
	\item L'impiegato dell'ufficio amministrazione seleziona il giorno per il quale vuole modificare la turnazione.
	\item L'impiegato dell'ufficio amministrazione seleziona il dipendente per il quale vuole modificare la turnazione.
	\item L'impiegato dell'ufficio amministrazione seleziona il nuovo turno da assegnare al dipendente.
	\item L'impiegato dell'ufficio amministrazione preme il pulsante "Genera".
		\begin{enumerate}
			\item Il sistema modifica in automatico le turnazioni.
			\item Il sistema mostra le turnazioni aggiornate all'impiegato dell'ufficio amministrazione.
		\end{enumerate}
	\item L'impiegato dell'ufficio amministrazione preme il pulsante "Conferma".
		\begin{enumerate}
			\item Le turnazioni aggiornate vengono inserite all'interno del database.
			\item Ciascun dipendente che ha subìto modifiche riceve una mail contenente le indicazioni sulle modifiche subìte ai propri orari.
		\end{enumerate}
\end{enumerate}
%------------------------------------------------
\paragraph{Alternative}
%------------------------------------------------
\begin{itemize}
	\item \underline{Selezione giorno errata}
		\begin{itemize}
			\item Al passo (2) l'impiegato dell'ufficio amministrazione seleziona una data non esistente o per la quale non è ancora stata generata una turnazione. Viene quindi mostrato un messaggio di errore.
		\end{itemize}
	\item \underline{Selezione dipendente errato}
		\begin{itemize}
			\item Al passo (3) l'impiegato dell'ufficio amministrazione seleziona un dipendente non esistente. Viene quindi mostrato un messaggio di errore.
		\end{itemize}
	\item \underline{Dipendenti non sufficienti}
		\begin{itemize}
			\item Al passo (4) il sistema non riesce a trovare un'allocazione ammissibile, poichè i dipendenti o la disponibilità di ore non sono sufficienti. Viene quindi mostrato un messaggio di errore.
		\end{itemize}
	\item \underline{Indirizzi e-mail non inseriti}
		\begin{itemize}
			\item Al passo (5) il sistema non trova gli indirizzi e-mail di alcuni dipendenti. Viene mostrata una finestra che comunica i dati mancanti.
		\end{itemize}
\end{itemize}
%------------------------------------------------
\paragraph{Estensioni}
%------------------------------------------------
\begin{itemize}
	\item \underline{Generazione nuova turnazione}
		\begin{itemize}
			\item Al passo (6) l'impiegato dell'ufficio amministrazione clicca su "Genera nuova turnazione". Si riparte quindi dal passo (5.a).
		\end{itemize}
\end{itemize}