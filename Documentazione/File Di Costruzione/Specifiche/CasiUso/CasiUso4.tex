%------------------------------------------------
\subsection{UC4: Inserimento nuova attività}
%------------------------------------------------
\paragraph{Descrizione}
%------------------------------------------------
Si vuole inserire una nuova attività.
%------------------------------------------------
\paragraph{Requisiti coperti}
%------------------------------------------------
DA01, DA05, DA06, DA07, DA08
%------------------------------------------------
\paragraph{Attori coinvolti}
%------------------------------------------------
Gestore del personale
%------------------------------------------------
\paragraph{Precondizioni}
%------------------------------------------------
L'attività non è presente nel database.
%------------------------------------------------
\paragraph{Postcondizioni}
%------------------------------------------------
L'attività viene inserita nel database.
%------------------------------------------------
\paragraph{Processo}
%------------------------------------------------
Di seguito è descritto il processo:
\begin{enumerate}
	\item Il gestore del personale preme il tasto “Aggiungi attività”.
	\item Il gestore del personale compila la form in cui sono richiesti i dati dell'attività (codice, nome, postazione, descrizione). 
	\item Il gestore del personale preme “OK”
		\begin{enumerate}
			\item I dati della nuova attività vengono inseriti nel database.
			\item Il sistema mostra un riepilogo indicante i dati delle attività attualmente presenti nel database.
		\end{enumerate}
\end{enumerate}
%------------------------------------------------
\paragraph{Alternative}
%------------------------------------------------
\begin{itemize}
	\item \underline{Inserimento dati errati}
		\begin{itemize}
			\item Al passo (2) sono inseriti dati errati, in questo caso l'inserimento è annullato e viene lanciato un messaggio di errore. 
			\item Con “dati errati” si intende codice, nome, postazione o descrizione vuoti.
		\end{itemize}
\end{itemize}
%------------------------------------------------
\paragraph{Estensioni}
%------------------------------------------------
\begin{itemize}
	\item \underline{Stampa dati di riepilogo}
		\begin{itemize}
			\item Al passo (3.b) il gestore del personale può cliccare sul pulsante “Stampa”, per stampare la scheda di riepilogo contenente i dati delle attività inserite. 
		\end{itemize}
\end{itemize}