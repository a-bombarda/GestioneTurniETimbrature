%------------------------------------------------
\subsection{UC2: Cancellazione dipendente}
%------------------------------------------------
\paragraph{Descrizione}
%------------------------------------------------
Si vuole eliminare dal database un dipendente.
%------------------------------------------------
\paragraph{Requisiti coperti}
%------------------------------------------------
DD02, DD05
%------------------------------------------------
\paragraph{Attori coinvolti}
%------------------------------------------------
Gestore del personale
%------------------------------------------------
\paragraph{Precondizioni}
%------------------------------------------------
Vari dipendenti sono presenti nel database.
%------------------------------------------------
\paragraph{Postcondizioni}
%------------------------------------------------
Il dipendente viene eliminato dal database.
%------------------------------------------------
\paragraph{Processo}
%------------------------------------------------
Di seguito è descritto il processo:
\begin{enumerate}
	\item Il gestore del personale clicca su “Lista dipendenti” nella pagina di gestione dei dipendenti.
		\begin{enumerate}
			\item Viene visualizzata la lista di tutti i dipendenti.
		\end{enumerate}
	\item Il gestore del personale seleziona il dipendente che vuole eliminare. 
	\item Il gestore del personale clicca su “Elimina dipendente”. 
		\begin{enumerate}
			\item E' richiesta la conferma dell’azione: se l’azione è confermata, il dipendente viene eliminato dal database.
			\item Viene aggiornata la lista di tutti i dipendenti, in modo che si possa eventualmente procedere ad una ulteriore eliminazione.	
		\end{enumerate}
\end{enumerate}
%------------------------------------------------
\paragraph{Alternative}
%------------------------------------------------
\begin{itemize}
	\item \underline{Dipendente con turnazione assegnata}
		\begin{itemize}
			\item Se al passo (3) si sta cercando di eliminare un dipendente assegnato ad uno o più turni per i giorni futuri, la cancellazione è annullata e viene visualizzato un messaggio di errore. 
		\end{itemize}
\end{itemize}