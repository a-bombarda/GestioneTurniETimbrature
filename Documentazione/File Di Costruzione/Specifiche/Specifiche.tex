\chapter{Specifiche}
%------------------------------------------------
\section{Introduzione}
%------------------------------------------------
\paragraph{Importanza delle funzioni}
Di seguito vengono riportate le varie funzioni richieste per soddisfare i requisiti posti dal committente. Non tutte le funzioni, però, sono importanti allo stesso livello, ma alcune richiedono di essere implementate prima di altre.\\
In particolare verrà definito un livello di importanza in base alla seguente classificazione:
\begin{itemize}
	\item \underline{Funzioni con alta priorità}: sono le funzioni sulle quali si basa il funzionamento "base" del programma che, quindi, richiedono di essere realizzate prima di quelle con media priorità.
	\item \underline{Funzioni con media priorità}: sono le funzioni che richiedono di essere realizzate prima di quelle con bassa priorità. Date le caratteristiche di queste funzioni è possibile implementarle parallelamente alle altre funzioni.
	\item \underline{Funzioni con bassa priorità}: sono le funzioni "di contorno", dalle quali non dipende alcuna altra funzione.	 
\end{itemize}
%------------------------------------------------
\section{Funzionalità richieste}
%------------------------------------------------
\subsection{Gestione dipendenti}
\setlength{\extrarowheight}{7pt}
\begin{longtable}{|c|c|c|}
\hline 
\textbf{Codice} & \textbf{Importanza} & \textbf{Nome funzione}
\tabularnewline
\hline
DD01 &	Alta	&	Inserimento dipendente\tabularnewline
DD02 &	Bassa	&	Cancellazione dipendente\tabularnewline
DD03 &	Media	&	Modifica dipendente\tabularnewline
DD04 &	Bassa	& 	Invio mail riepilogo dati\tabularnewline
DD05 &	Bassa	&	Visualizzazione dati\tabularnewline
DD06 &	Bassa 	& 	Stampa dati\tabularnewline
DD07 &	Bassa	&	Statistiche dipendenti\tabularnewline
DD08 &	Alta	&	Inserimento dati nel database\tabularnewline
\hline
\caption{Funzionalità per la gestione dei dipendenti}
\label{tab:GestioneDipendenti}
\end{longtable}
%------------------------------------------------
\subsection{Gestione dati base dell'azienda}
\setlength{\extrarowheight}{7pt}
\begin{longtable}{|c|c|c|}
\hline 
\textbf{Codice} & \textbf{Importanza} & \textbf{Nome funzione}
\tabularnewline
\hline
DA01 &	Alta 	&	Inserimento attività\tabularnewline
DA02 &	Alta 	&	Cancellazione attività\tabularnewline
DA03 &	Media 	&	Modifica attività\tabularnewline
DA04 &	Alta	&	Modifica numero dipendenti per attività\tabularnewline
DA05 &	Bassa	&	Statistiche azienda\tabularnewline
DA06 &	Bassa	&	Visualizzazione dati azienda\tabularnewline
DA07 &	Bassa	&	Stampa dati azienda\tabularnewline
DA08 &	Alta	&	Inserimento dati nel database\tabularnewline
\hline
\caption{Funzionalità per la gestione dei dati base dell'azienda}
\label{tab:GestioneDatiBaseAzienda}
\end{longtable}
%------------------------------------------------
\subsection{Gestione turni dipendenti}
\setlength{\extrarowheight}{7pt}
\begin{longtable}{|c|c|c|}
\hline 
\textbf{Codice} & \textbf{Importanza} & \textbf{Nome funzione}
\tabularnewline
\hline
TD01 &  Bassa &	Visualizzazione turnazione settimanale\tabularnewline
TD02 &	Media &	Visualizzazione turnazione giornaliera\tabularnewline
TD03 &	Bassa &	Stampa report turnazione settimanale\tabularnewline
TD04 &	Bassa &	Stampa report turnazione giornaliera\tabularnewline
TD05 &	Bassa &	Invio mail turnazione dipendente\tabularnewline
TD06 &	Alta	 &	Generazione turnazione settimanale\tabularnewline
TD07 &	Alta	 &	Generazione turnazione giornaliera\tabularnewline
TD08 &	Alta	 &	Assegnazione dipendente ad un turno\tabularnewline
TD09 &	Alta	 &  Richiesta cambio turno dipendente\tabularnewline
TD10 &	Alta	 &	Ricalcolo turnazione dopo cambio turno\tabularnewline
TD11 &	Bassa &	Invia mail cambio turno\tabularnewline
TD12 &	Media &	Allarme turnazione non possibile\tabularnewline
TD13 &  Bassa &	Visualizzazione turnazione settimana dipendente\tabularnewline
TD14 &	Media &	Visualizzazione turnazione giorno dipendente\tabularnewline
TD15 & 	Media & Salvataggio dati turnazioni in database\tabularnewline
\hline
\caption{Funzionalità per la gestione dei turni dei dipendenti}
\label{tab:GestioneTurniDipendenti}
\end{longtable}
%------------------------------------------------
\subsection{Gestione file XML}
\setlength{\extrarowheight}{7pt}
\begin{longtable}{|c|c|c|}
\hline 
\textbf{Codice} & \textbf{Importanza} & \textbf{Nome funzione}
\tabularnewline
\hline
FM00 &	Alta 	&	Parsing file XML\tabularnewline
FM01 &	Media 	&	Estrazione e selezione campi\tabularnewline
FM02 &	Alta 	&	Memorizzazione timbrature nel database\tabularnewline
FM03 &	Bassa	&	Modifica timbrature dal database\tabularnewline
FM04 &	Bassa	&	Eliminazione timbrature dal database\tabularnewline
FM05 &	Media	&	Visualizzazione dei dati grezzi\tabularnewline
\hline
\caption{Funzionalità per la gestione dei file XML}
\label{tab:GestioneXML}
\end{longtable}
%------------------------------------------------
\subsection{Gestione orari}
\setlength{\extrarowheight}{7pt}
\begin{longtable}{|c|c|c|}
\hline 
\textbf{Codice} & \textbf{Importanza} & \textbf{Nome funzione}
\tabularnewline
\hline
SM00 & Alta & Calcolo giornaliero ore lavoro ordinario\tabularnewline
SM01 & Alta & Calcolo giornaliero ore lavoro straordinario\tabularnewline
SM02 & Alta & Calcolo giornaliero ore ferie/permessi\tabularnewline
SM03 & Bassa & Revisione e controllo generale\tabularnewline
SM04 & Alta & Calcolo giornaliero ore malattia\tabularnewline
SM05 & Media & Calcolo mensile ore lavoro ordinario\tabularnewline
SM06 & Media & Calcolo mensile ore lavoro straordinario\tabularnewline
SM07 & Media & Calcolo mensile ore ferie/permessi\tabularnewline
SM08 & Media & Calcolo mensile ore malattia\tabularnewline
SM09 & Alta & Inserimento dati giornalieri nel database\tabularnewline
SM10 & Alta & Inserimento dati mensili nel database\tabularnewline
SM11 & Media & Applicazione incremento percentuale\tabularnewline
SM12 & Bassa & Calcolo stipendio mensile\tabularnewline
SM13 & Bassa & Generazione report mensile dipendente\tabularnewline
SM14 & Alta & Inserimento malattia/ferie\tabularnewline
SM15 & Alta & Aggiornamento malattia/ferie\tabularnewline
SM16 & Alta & Eliminazione malattia/ferie\tabularnewline
SM17 & Bassa & Inserimento ore straordinarie necessarie\tabularnewline
SM18 & Bassa & Aggiornamento ore straordinarie necessarie\tabularnewline
SM19 & Bassa & Eliminazione ore straordinarie necessarie\tabularnewline
SM20 & Alta & Modifica dati giornalieri nel database\tabularnewline
SM21 & Alta & Eliminazione dati giornalieri nel database\tabularnewline
SM22 & Media & Modifica dati mensili nel database\tabularnewline
SM23 & Media & Eliminazione dati mensili nel database\tabularnewline
SM24 & Media & Visualizzazione dati mensili dipendente\tabularnewline
SM25 & Bassa & Verifica copertura turno assegnato\tabularnewline
SM26 & Media & Visualizzazione dati straordinari\tabularnewline
SM27 & Media & Visualizzazione dati giornalieri dipendente\tabularnewline
SM28 & Bassa & Prenotazione per ore di straordinario\tabularnewline
\hline
\caption{Funzionalità per la gestione degli orari dei dipendenti}
\label{tab:GestioneOrari}
\end{longtable}
%------------------------------------------------
\section{Analisi casi d'uso}
%------------------------------------------------
%------------------------------------------------
\subsection{UC1: Inserimento dipendente}
%------------------------------------------------
\paragraph{Descrizione}
%------------------------------------------------
Si vuole inserire un nuovo dipendente.
%------------------------------------------------
\paragraph{Requisiti coperti}
%------------------------------------------------
DD01, DD03-DD08, DA01, DA08
%------------------------------------------------
\paragraph{Attori coinvolti}
%------------------------------------------------
Gestore del personale
%------------------------------------------------
\paragraph{Precondizioni}
%------------------------------------------------
Il dipendente non è presente nel database.
%------------------------------------------------
\paragraph{Postcondizioni}
%------------------------------------------------
Il dipendente è stato aggiunto al database.
%------------------------------------------------
\paragraph{Processo}
%------------------------------------------------
Di seguito è descritto il processo:
\begin{enumerate}
	\item Il gestore del personale preme il tasto “Aggiungi dipendente”.
	\item Il gestore del personale compila la form in cui sono richiesti i dati del dipendente (nome, cognome, data di nascita, data di inizio e fine contratto, indirizzo email, numero di ore contrattuali a settimana, attività svolte, numero del badge, etc.). 	
	\item Il gestore del personale preme “OK” per confermare l’inserimento del dipendente.
		\begin{enumerate}
			\item Il sistema mostra il riepilogo dei dati inseriti per il dipendente.
		\end{enumerate}
	\item Il gestore del personale preme “Conferma” per confermare l’inserimento del dipendente.
		\begin{enumerate}
			\item I dati del dipendente vengono inseriti all’interno del database.
			\item Il dipendente riceve una e-mail contenente la conferma di inserimento ed il riepilogo delle informazioni inserite.
			\item Il sistema mostra un riepilogo indicante, per ogni attività disponibile, quali e quanti dipendenti sono disponibili.
		\end{enumerate}
\end{enumerate}
%------------------------------------------------
\paragraph{Alternative}
%------------------------------------------------
\begin{itemize}
	\item \underline{Inserimento dati errati}
		\begin{itemize}
			\item Al passo (2) sono inseriti dati errati, in questo caso l’inserimento è annullato e viene lanciato un messaggio di errore. 
			\item In alternativa ci si può accorgere di aver inserito dati errati al passo (3). In questo caso il gestore del personale può cliccare su “Modifica dati”, tornando, quindi, al passo (2).
			\item Con “dati errati” si intende nome, cognome, posizione vuoti o formato delle date diverso da gg/mm/aaaa. 
		\end{itemize}
	\item \underline{Dipendente a tempo indeterminato}
		\begin{itemize}
			\item Al passo (2) non viene compilato il campo “data fine contratto”. 
			\item Quando si clicca “Prosegui” viene lanciato un messaggio di avviso per informare del campo vuoto, ma è comunque possibile proseguire. 
		\end{itemize}
	\item \underline{Attività non presente}
		\begin{itemize}
			\item Al passo (2) il gestore del personale non trova l'attività da assegnare al dipendente.
			\item Il gestore del personale clicca sul pulsante “Nuova Attività”.
				\begin{itemize}
					\item Da qua si procede come da UC4.
				\end{itemize}
		\end{itemize}
		\item \underline{Dipendente senza indirizzo e-mail}
		\begin{itemize}
			\item Al passo (2) non viene compilato il campo “indirizzo e-mail”. 
			\item Quando si clicca “Prosegui” viene lanciato un messaggio di avviso per informare del campo vuoto, ma è comunque possibile proseguire. 
		\end{itemize}
\end{itemize}
%------------------------------------------------
\paragraph{Estensioni}
%------------------------------------------------
\begin{itemize}
	\item \underline{Stampa dati dipendente}
		\begin{itemize}
			\item Al passo (3) il gestore del personale può cliccare sul pulsante “Stampa”, per stampare la scheda di riepilogo contenente i dati inseriti per il nuovo dipendente.
		\end{itemize}
\end{itemize}
%------------------------------------------------
\subsection{UC2: Cancellazione dipendente}
%------------------------------------------------
\paragraph{Descrizione}
%------------------------------------------------
Si vuole eliminare dal database un dipendente.
%------------------------------------------------
\paragraph{Requisiti coperti}
%------------------------------------------------
DD02, DD05
%------------------------------------------------
\paragraph{Attori coinvolti}
%------------------------------------------------
Gestore del personale
%------------------------------------------------
\paragraph{Precondizioni}
%------------------------------------------------
Vari dipendenti sono presenti nel database.
%------------------------------------------------
\paragraph{Postcondizioni}
%------------------------------------------------
Il dipendente viene eliminato dal database.
%------------------------------------------------
\paragraph{Processo}
%------------------------------------------------
Di seguito è descritto il processo:
\begin{enumerate}
	\item Il gestore del personale clicca su “Lista dipendenti” nella pagina di gestione dei dipendenti.
		\begin{enumerate}
			\item Viene visualizzata la lista di tutti i dipendenti.
		\end{enumerate}
	\item Il gestore del personale seleziona il dipendente che vuole eliminare. 
	\item Il gestore del personale clicca su “Elimina dipendente”. 
		\begin{enumerate}
			\item E' richiesta la conferma dell’azione: se l’azione è confermata, il dipendente viene eliminato dal database.
			\item Viene aggiornata la lista di tutti i dipendenti, in modo che si possa eventualmente procedere ad una ulteriore eliminazione.	
		\end{enumerate}
\end{enumerate}
%------------------------------------------------
\paragraph{Alternative}
%------------------------------------------------
\begin{itemize}
	\item \underline{Dipendente con turnazione assegnata}
		\begin{itemize}
			\item Se al passo (3) si sta cercando di eliminare un dipendente assegnato ad uno o più turni per i giorni futuri, la cancellazione è annullata e viene visualizzato un messaggio di errore. 
		\end{itemize}
\end{itemize}
%------------------------------------------------
\subsection{UC3: Modifica dipendente}
%------------------------------------------------
\paragraph{Descrizione}
%------------------------------------------------
Si vogliono modificare alcuni dati di un dipendente.
%------------------------------------------------
\paragraph{Requisiti coperti}
%------------------------------------------------
DD03-DD08, DA01, DA08
%------------------------------------------------
\paragraph{Attori coinvolti}
%------------------------------------------------
Gestore del personale
%------------------------------------------------
\paragraph{Precondizioni}
%------------------------------------------------
Vari utenti sono presenti nel database.
%------------------------------------------------
\paragraph{Postcondizioni}
%------------------------------------------------
Alcuni dati del dipendente scelto sono modificati.
%------------------------------------------------
\paragraph{Processo}
%------------------------------------------------
Di seguito è descritto il processo:
\begin{enumerate}
	\item Il gestore del personale clicca su “Lista dipendenti” nella pagina di gestione dei dipendenti.
		\begin{enumerate}
			\item Viene visualizzata la lista di tutti i dipendenti.
		\end{enumerate}
	\item Il gestore del personale seleziona il dipendente che vuole modificare.	
	\item Il gestore del personale clicca su “Modifica dipendente”. 
		\begin{enumerate}
			\item Viene mostrata la form per la modifica dei dati.
			\item Il gestore del personale modifica i dati.
		\end{enumerate}
	\item Il gestore del personale clicca su “Conferma modifica”.
		\begin{enumerate}
			\item E' richiesta la conferma dell’azione: se l’azione è confermata, il dipendente è modificato.
			\item I nuovi dati del dipendente vengono salvati all’interno del database.
			\item Il dipendente riceve una e-mail contenente la conferma di inserimento ed il riepilogo delle informazioni inserite.
			\item Il sistema mostra un riepilogo indicante, per ogni attività disponibile, quali e quanti dipendenti sono disponibili.
			\item Viene aggiornata la lista di tutti i dipendenti, in modo che si possa eventualmente procedere ad una ulteriore modifica.
		\end{enumerate}
\end{enumerate}
%------------------------------------------------
\paragraph{Alternative}
%------------------------------------------------
\begin{itemize}
	\item \underline{Inserimento dati errati}
		\begin{itemize}
			\item Al passo (3.b) sono inseriti dati errati, in questo caso la modifica è annullata e viene lanciato un messaggio di errore. 
			\item Con “dati errati” si intende nome, cognome, posizione vuoti o formato delle date diverso da gg/mm/aaaa. 
		\end{itemize}
	\item \underline{Dipendente a tempo indeterminato}
		\begin{itemize}
			\item Al passo (3.b) non viene compilato il campo “data fine contratto”. 
			\item Quando si clicca “Conferma modifica” viene lanciato un messaggio di avviso per informare del campo vuoto, ma è comunque possibile proseguire. 
		\end{itemize}
	\item \underline{Attività non presente}
		\begin{itemize}
			\item Al passo (3.b) il gestore del personale non trova l'attività da assegnare al dipendente.
			\item Il gestore del personale clicca sul pulsante “Nuova Attività”.
				\begin{itemize}
					\item Da qua si procede come da UC4.
				\end{itemize}
		\end{itemize}
		\item \underline{Dipendente senza indirizzo e-mail}
		\begin{itemize}
			\item Al passo (3.b) non viene compilato il campo “indirizzo e-mail”. 
			\item Quando si clicca “Prosegui” viene lanciato un messaggio di avviso per informare del campo vuoto, ma è comunque possibile proseguire. 
		\end{itemize}
\end{itemize}
%------------------------------------------------
\subsection{UC4: Inserimento nuova attività}
%------------------------------------------------
\paragraph{Descrizione}
%------------------------------------------------
Si vuole inserire una nuova attività.
%------------------------------------------------
\paragraph{Requisiti coperti}
%------------------------------------------------
DA01, DA05, DA06, DA07, DA08
%------------------------------------------------
\paragraph{Attori coinvolti}
%------------------------------------------------
Gestore del personale
%------------------------------------------------
\paragraph{Precondizioni}
%------------------------------------------------
L'attività non è presente nel database.
%------------------------------------------------
\paragraph{Postcondizioni}
%------------------------------------------------
L'attività viene inserita nel database.
%------------------------------------------------
\paragraph{Processo}
%------------------------------------------------
Di seguito è descritto il processo:
\begin{enumerate}
	\item Il gestore del personale preme il tasto “Aggiungi attività”.
	\item Il gestore del personale compila la form in cui sono richiesti i dati dell'attività (codice, nome, postazione, descrizione). 
	\item Il gestore del personale preme “OK”
		\begin{enumerate}
			\item I dati della nuova attività vengono inseriti nel database.
			\item Il sistema mostra un riepilogo indicante i dati delle attività attualmente presenti nel database.
		\end{enumerate}
\end{enumerate}
%------------------------------------------------
\paragraph{Alternative}
%------------------------------------------------
\begin{itemize}
	\item \underline{Inserimento dati errati}
		\begin{itemize}
			\item Al passo (2) sono inseriti dati errati, in questo caso l'inserimento è annullato e viene lanciato un messaggio di errore. 
			\item Con “dati errati” si intende codice, nome, postazione o descrizione vuoti.
		\end{itemize}
\end{itemize}
%------------------------------------------------
\paragraph{Estensioni}
%------------------------------------------------
\begin{itemize}
	\item \underline{Stampa dati di riepilogo}
		\begin{itemize}
			\item Al passo (3.b) il gestore del personale può cliccare sul pulsante “Stampa”, per stampare la scheda di riepilogo contenente i dati delle attività inserite. 
		\end{itemize}
\end{itemize}
%------------------------------------------------
\subsection{UC5: Cancellazione di una attività}
%------------------------------------------------
\paragraph{Descrizione}
%------------------------------------------------
Si vuole eliminare una attività precedentemente inserita.
%------------------------------------------------
\paragraph{Requisiti coperti}
%------------------------------------------------
DA02, DA05, DA06
%------------------------------------------------
\paragraph{Attori coinvolti}
%------------------------------------------------
Gestore del personale
%------------------------------------------------
\paragraph{Precondizioni}
%------------------------------------------------
Sono presenti varie attività nel database.
%------------------------------------------------
\paragraph{Postcondizioni}
%------------------------------------------------
L'attività viene cancellata dal database.
%------------------------------------------------
\paragraph{Processo}
%------------------------------------------------
Di seguito è descritto il processo:
\begin{enumerate}
	\item Il gestore del personale preme il tasto “Lista attività”.
		\begin{enumerate}
			\item Viene visualizzata la lista di tutte le attività.
		\end{enumerate}
	\item Il gestore del personale seleziona una attività. 
	\item Il gestore del personale clicca su "Elimina attività".
		\begin{enumerate}
			\item E' richiesta la conferma dell’azione: se l’azione è confermata, l'attività è eliminata.
			\item Viene aggiornata la lista di tutte le attività, in modo che si possa eventualmente procedere ad una ulteriore eliminazione.
		\end{enumerate}
\end{enumerate}
%------------------------------------------------
\paragraph{Alternative}
%------------------------------------------------
\begin{itemize}
	\item \underline{Attività assegnata a dipendenti}
		\begin{itemize}
			\item Al passo (3) si sta cercando di eliminare una attività assegnata ancora da almeno un dipendente, la cancellazione è annullata e viene visualizzato un messaggio di errore.
			\item In questa situazione sarà necessario, in via preventiva, rimuovere l'attività dalla lista di quelle ricoperte dal dipendente, come da UC3.
		\end{itemize}
\end{itemize}

%------------------------------------------------
\subsection{UC6: Modifica di una attività}
%------------------------------------------------
\paragraph{Descrizione}
%------------------------------------------------
Si vuole modificare un'attività precedentemente inserita.
%------------------------------------------------
\paragraph{Requisiti coperti}
%------------------------------------------------
DA03, DA05-DA08
%------------------------------------------------
\paragraph{Attori coinvolti}
%------------------------------------------------
Gestore del personale
%------------------------------------------------
\paragraph{Precondizioni}
%------------------------------------------------
Sono presenti varie attività nel database.
%------------------------------------------------
\paragraph{Postcondizioni}
%------------------------------------------------
L'attività viene modificata nel database.
%------------------------------------------------
\paragraph{Processo}
%------------------------------------------------
Di seguito è descritto il processo:
\begin{enumerate}
	\item Il gestore del personale preme il tasto “Lista attività”.
		\begin{enumerate}
			\item Viene visualizzata la lista di tutte le attività.
		\end{enumerate}
	\item Il gestore del personale seleziona una attività. 
	\item Il gestore del personale clicca su "Modifica attività".
		\begin{enumerate}
			\item Viene mostrata una form nella quale è possibile modificare i vari dati.
			\item Il gestore del personale modifica i dati.
		\end{enumerate}
	\item Il gestore del personale clicca su "Conferma modifica".
		\begin{enumerate}
			\item E' richiesta la conferma dell’azione: se l’azione è confermata, l'attività è modificata.
			\item I nuovi dati dell'attività sono inseriti all’interno del database.
			\item Il sistema mostra un riepilogo indicante i dati delle attività attualmente presenti nel database.
		\end{enumerate}
\end{enumerate}
%------------------------------------------------
\paragraph{Alternative}
%------------------------------------------------
\begin{itemize}
	\item \underline{Inserimento dati errati}
		\begin{itemize}
			\item Al passo (3) sono inseriti dati errati, in questo caso l’inserimento è annullato e viene lanciato un messaggio di errore. 
			\item Con “dati errati” si intende codice, nome, postazione o descrizione vuoti.
		\end{itemize}
\end{itemize}
%------------------------------------------------
\paragraph{Estensioni}
%------------------------------------------------
\begin{itemize}
	\item \underline{Stampa dati di riepilogo}
		\begin{itemize}
			\item Al passo (4.c), il gestore del personale può cliccare sul pulsante “Stampa”, per stampare la scheda di riepilogo contenente i dati delle attività inserite.  
		\end{itemize}
\end{itemize}

%------------------------------------------------
\subsection{UC7. Allocazione dipendenti per ogni attività}
%------------------------------------------------
\paragraph{Descrizione}
%------------------------------------------------
Si vuole impostare il numero di dipendenti necessari per ogni attività in un intervallo di date.
%------------------------------------------------
\paragraph{Requisiti coperti}
%------------------------------------------------
DA04-DA08
%------------------------------------------------
\paragraph{Attori coinvolti}
%------------------------------------------------
Gestore del personale
%------------------------------------------------
\paragraph{Precondizioni}
%------------------------------------------------
Sono presenti varie attività nel database.
%------------------------------------------------
\paragraph{Postcondizioni}
%------------------------------------------------
Il numero impostato di dipendenti necessari per ogni attività viene inserito all’interno del database.
%------------------------------------------------
\paragraph{Processo}
%------------------------------------------------
Di seguito è descritto il processo:
\begin{enumerate}
	\item Il gestore del personale preme il tasto “Imposta richiesta manodopera per attività”.
	\item Il gestore del personale seleziona l’intervallo di date all’interno del quale vuole impostare la richiesta di manodopera necessaria per attività. 
	\item Il gestore del personale seleziona l'attività.
	\item Il gestore del personale seleziona il turno per il quale vuole impostare la richiesta di manodopera.
	\item Il gestore del personale seleziona il numero di dipendenti, che eseguono l'attività richiesta, necessari per il turno selezionato.
	\item Il gestore del personale clicca sul pulsante “Conferma”.
		\begin{enumerate}
			\item I nuovi dati vengono inseriti all’interno del database.
			\item Il sistema mostra un riepilogo indicante i dati delle attività attualmente presenti nel database, con relative richieste di dipendenti, per l’intervallo di date selezionato.
		\end{enumerate}
\end{enumerate}
%------------------------------------------------
\paragraph{Alternative}
%------------------------------------------------
\begin{itemize}
	\item \underline{Inserimento data errata}
		\begin{itemize}
			\item Al passo (2) il gestore del personale inserisce delle date non valide o con formato non corretto. Viene mostrato un messaggio di errore e viene data la possibilità di modificare le due date.
		\end{itemize}
	\item \underline{Inserimento numero dipendenti errato}
		\begin{itemize}
			\item Al passo (5) il gestore del personale inserisce un valore numerico negativo. Viene mostrato un messaggio di errore e viene data la possibilità di modificare il valore.
		\end{itemize}
\end{itemize}
%------------------------------------------------
\subsection{UC8: Generazione turnazione settimanale}
%------------------------------------------------
\paragraph{Descrizione}
%------------------------------------------------
Si vuole generare la turnazione di una nuova settimana, includendo tutti i dipendenti che non sono in ferie.\\
L'assegnazione dei turni deve rispettare il numero massimo di ore settimanali di ciascun dipendente.
%------------------------------------------------
\paragraph{Requisiti coperti}
%------------------------------------------------
TD01, TD02, TD05, TD06, TD07, TD08,  TD12, TD15, DA04-DA08
%------------------------------------------------
\paragraph{Attori coinvolti}
%------------------------------------------------
Impiegato dell'ufficio amministrazione
%------------------------------------------------
\paragraph{Precondizioni}
%------------------------------------------------
\begin{itemize}
	\item I dipendenti sono inseriti all'interno del database del programma.
	\item Non è ancora stata generata una turnazione per la settimana.
\end{itemize}
%------------------------------------------------
\paragraph{Postcondizioni}
%------------------------------------------------
Turni assegnati a tutti i dipendenti per la settimana richiesta, mail (contenente il resoconto della turnazione generata) inviata a ciascun dipendente.
%------------------------------------------------
\paragraph{Processo}
%------------------------------------------------
Di seguito è descritto il processo:
\begin{enumerate}
	\item L'impiegato dell'ufficio amministrazione preme il pulsante "Genera turnazione settimanale".
	\item L'impiegato dell'ufficio amministrazione seleziona la settimana per la quale vuole generare la turnazione.
	\item L'impiegato dell'ufficio amministrazione preme il pulsante "Genera".
		\begin{enumerate}
			\item Il sistema genera in automatico le turnazioni.
			\item Il sistema mostra le turnazioni generate all'impiegato dell'ufficio amministrazione.
		\end{enumerate}
	\item L'impiegato dell'ufficio amministrazione preme il pulsante "Conferma"
		\begin{enumerate}
			\item Le turnazioni assegnate vengono inserite all'interno del database.
			\item Ciascun dipendente riceve una mail contenente le indicazioni sui turni che gli sono stati assegnati per la settimana.
		\end{enumerate}
\end{enumerate}
%------------------------------------------------
\paragraph{Alternative}
%------------------------------------------------
\begin{itemize}
	\item \underline{Selezione settimana errata}
		\begin{itemize}
			\item Al passo (2) l'impiegato dell'ufficio amministrazione seleziona una settimana non esistente o per la quale è già stata assegnata una turnazione. Viene quindi mostrato un messaggio di errore.
		\end{itemize}
	\item \underline{Dipendenti non sufficienti}
		\begin{itemize}
			\item Al passo (3) il sistema non riesce a trovare un'allocazione ammissibile, poichè i dipendenti o la disponibilità di ore non sono sufficienti. Viene quindi mostrato un messaggio di errore.
		\end{itemize}
	\item \underline{Indirizzi e-mail non inseriti}
		\begin{itemize}
			\item Al passo (4) il sistema non trova gli indirizzi e-mail di alcuni dipendenti. Viene mostrata una finestra che comunica i dati mancanti.
		\end{itemize}
\end{itemize}
%------------------------------------------------
\paragraph{Estensioni}
%------------------------------------------------
\begin{itemize}
	\item \underline{Generazione nuova turnazione}
		\begin{itemize}
			\item Al passo (4) l'impiegato dell'ufficio amministrazione clicca su "Genera nuova turnazione". Si riparte quindi dal passo (3.a).
		\end{itemize}
\end{itemize}
%------------------------------------------------
\subsection{UC9: Visualizzazione turnazione settimanale}
%------------------------------------------------
\paragraph{Descrizione}
%------------------------------------------------
Si vuole visualizzare la turnazione assegnata per una specifica settimana.
%------------------------------------------------
\paragraph{Requisiti coperti}
%------------------------------------------------
TD01, TD02, TD03, TD05, TD13, TD14
%------------------------------------------------
\paragraph{Attori coinvolti}
%------------------------------------------------
Impiegato dell'ufficio amministrazione
%------------------------------------------------
\paragraph{Precondizioni}
%------------------------------------------------
La turnazione della settimana è inserita all'interno del database, quindi è già stata generata come da UC8.
%------------------------------------------------
\paragraph{Postcondizioni}
%------------------------------------------------
La turnazione della settimana viene mostrata all'impiegato dell'ufficio amministrazione
%------------------------------------------------
\paragraph{Processo}
%------------------------------------------------
Di seguito è descritto il processo:
\begin{enumerate}
	\item L'impiegato dell'ufficio amministrazione preme il pulsante "Visualizza turnazione settimanale".
	\item L'impiegato dell'ufficio amministrazione seleziona la settimana per la quale vuole visualizzare la turnazione.
	\item L'impiegato dell'ufficio amministrazione preme il pulsante "Conferma".
		\begin{enumerate}
			\item La turnazione della settimana selezionata viene mostrata.
		\end{enumerate}
\end{enumerate}
%------------------------------------------------
\paragraph{Alternative}
%------------------------------------------------
\begin{itemize}
	\item \underline{Selezione settimana errata}
		\begin{itemize}
			\item Al passo (2) l'impiegato dell'ufficio amministrazione seleziona una settimana non esistente o per la quale non è ancora stata generata una turnazione. Viene quindi mostrato un messaggio di errore.
		\end{itemize}
\end{itemize}
%------------------------------------------------
\paragraph{Estensioni}
%------------------------------------------------
\begin{itemize}
	\item \underline{Visualizzazione turni del singolo dipendente}
		\begin{itemize}
			\item Al passo (2) l'impiegato dell'ufficio amministrazione può anche selezionare il dipendente del quale vuole conoscere la turnazione.\\
				Al passo (3), quindi, non verranno mostrati tutti gli orari, ma solamente quelli del dipendente selezionato.
		\end{itemize}
	\item \underline{Invio mail aggiornamento}
		\begin{itemize}
			\item Al passo (3) l'impiegato dell'ufficio amministrazione può anche cliccare sul pulsante "Invia mail". Il sistema si occuperà, quindi, di inviare una mail a tutti i dipendenti contenente i relativi turni.
		\end{itemize}
	\item \underline{Stampa report allocazione dipendenti sui turni}
		\begin{itemize}
			\item Al passo (3) l'impiegato dell'ufficio amministrazione può anche cliccare sul pulsante "Stampa turnazione". Il sistema si occuperà, quindi, di esportare in formato PDF la tabella con l'allocazione dei dipendenti sui turni, per la settimana selezionata.
		\end{itemize}
\end{itemize}
%------------------------------------------------
\subsection{UC10: Visualizzazione turnazione giornaliera}
%------------------------------------------------
\paragraph{Descrizione}
%------------------------------------------------
Si vuole visualizzare la turnazione assegnata per uno specifico giorno.
%------------------------------------------------
\paragraph{Requisiti coperti}
%------------------------------------------------
TD02, TD04, TD05, TD14
%------------------------------------------------
\paragraph{Attori coinvolti}
%------------------------------------------------
Impiegato dell'ufficio amministrazione
%------------------------------------------------
\paragraph{Precondizioni}
%------------------------------------------------
La turnazione del giorno è inserita all'interno del database, quindi è già stata generata come da UC1.
%------------------------------------------------
\paragraph{Postcondizioni}
%------------------------------------------------
La turnazione del giorno viene mostrata all'impiegato dell'ufficio amministrazione
%------------------------------------------------
\paragraph{Processo}
%------------------------------------------------
Di seguito è descritto il processo:
\begin{enumerate}
	\item L'impiegato dell'ufficio amministrazione preme il pulsante "Visualizza turnazione giornaliera".
	\item L'impiegato dell'ufficio amministrazione seleziona il giorno per il quale vuole visualizzare la turnazione.
	\item L'impiegato dell'ufficio amministrazione preme il pulsante "Conferma".
		\begin{enumerate}
			\item La turnazione del giorno selezionato viene mostrata.
		\end{enumerate}
\end{enumerate}
%------------------------------------------------
\paragraph{Alternative}
%------------------------------------------------
\begin{itemize}
	\item \underline{Selezione giorno errata}
		\begin{itemize}
			\item Al passo (2) l'impiegato dell'ufficio amministrazione seleziona una data non esistente o per la quale non è ancora stata generata una turnazione. Viene quindi mostrato un messaggio di errore.
		\end{itemize}
\end{itemize}
%------------------------------------------------
\paragraph{Estensioni}
%------------------------------------------------
\begin{itemize}
	\item \underline{Visualizzazione turni del singolo dipendente}
		\begin{itemize}
			\item Al passo (2) l'impiegato dell'ufficio amministrazione può anche selezionare il dipendente del quale vuole conoscere la turnazione.\\
				Al passo (3), quindi, non verranno mostrati tutti gli orari, ma solamente quelli del dipendente selezionato.
		\end{itemize}
	\item \underline{Invio mail aggiornamento}
		\begin{itemize}
			\item Al passo (3) l'impiegato dell'ufficio amministrazione può anche cliccare sul pulsante "Invia mail". Il sistema si occuperà, quindi, di inviare una mail a tutti i dipendenti contenenti i relativi turni per il giorno visualizzato.
		\end{itemize}
	\item \underline{Stampa report allocazione dipendenti sui turni}
		\begin{itemize}
			\item Al passo (3) l'impiegato dell'ufficio amministrazione può anche cliccare sul pulsante "Stampa turnazione". Il sistema si occuperà, quindi, di esportare in formato PDF la tabella con l'allocazione dei dipendenti sui turni, per il giorno selezionato.
		\end{itemize}
\end{itemize}
%------------------------------------------------
\subsection{UC11: Cambio turno dipendente}
%------------------------------------------------
\paragraph{Descrizione}
%------------------------------------------------
Si vuole, una volta ricevuta una specifica richiesta da un dipendente, modificare il relativo turno. La modifica di questo turno dovrà comportare, eventualmente, la modifica del turno di altri dipendenti.
%------------------------------------------------
\paragraph{Requisiti coperti}
%------------------------------------------------
TD01, TD02, TD09, TD10, TD11, TD12, TD15
%------------------------------------------------
\paragraph{Attori coinvolti}
%------------------------------------------------
Impiegato dell'ufficio amministrazione
%------------------------------------------------
\paragraph{Precondizioni}
%------------------------------------------------
\begin{itemize}
	\item I dipendenti sono inseriti all'interno del database del programma.
	\item E' già stata generata una turnazione per la giornata richiesta.
\end{itemize}
%------------------------------------------------
\paragraph{Postcondizioni}
%------------------------------------------------
\begin{itemize}
	\item Turni modificati, generando un assegnamento consistente.
	\item Mail contenente i nuovi turni inviata ai dipendenti che hanno subìto variazioni.
\end{itemize}
%------------------------------------------------
\paragraph{Processo}
%------------------------------------------------
Di seguito è descritto il processo:
\begin{enumerate}
	\item L'impiegato dell'ufficio amministrazione preme il pulsante "Modifica turno dipendente".
	\item L'impiegato dell'ufficio amministrazione seleziona il giorno per il quale vuole modificare la turnazione.
	\item L'impiegato dell'ufficio amministrazione seleziona il dipendente per il quale vuole modificare la turnazione.
	\item L'impiegato dell'ufficio amministrazione seleziona il nuovo turno da assegnare al dipendente.
	\item L'impiegato dell'ufficio amministrazione preme il pulsante "Genera".
		\begin{enumerate}
			\item Il sistema modifica in automatico le turnazioni.
			\item Il sistema mostra le turnazioni aggiornate all'impiegato dell'ufficio amministrazione.
		\end{enumerate}
	\item L'impiegato dell'ufficio amministrazione preme il pulsante "Conferma".
		\begin{enumerate}
			\item Le turnazioni aggiornate vengono inserite all'interno del database.
			\item Ciascun dipendente che ha subìto modifiche riceve una mail contenente le indicazioni sulle modifiche subìte ai propri orari.
		\end{enumerate}
\end{enumerate}
%------------------------------------------------
\paragraph{Alternative}
%------------------------------------------------
\begin{itemize}
	\item \underline{Selezione giorno errata}
		\begin{itemize}
			\item Al passo (2) l'impiegato dell'ufficio amministrazione seleziona una data non esistente o per la quale non è ancora stata generata una turnazione. Viene quindi mostrato un messaggio di errore.
		\end{itemize}
	\item \underline{Selezione dipendente errato}
		\begin{itemize}
			\item Al passo (3) l'impiegato dell'ufficio amministrazione seleziona un dipendente non esistente. Viene quindi mostrato un messaggio di errore.
		\end{itemize}
	\item \underline{Dipendenti non sufficienti}
		\begin{itemize}
			\item Al passo (4) il sistema non riesce a trovare un'allocazione ammissibile, poichè i dipendenti o la disponibilità di ore non sono sufficienti. Viene quindi mostrato un messaggio di errore.
		\end{itemize}
	\item \underline{Indirizzi e-mail non inseriti}
		\begin{itemize}
			\item Al passo (5) il sistema non trova gli indirizzi e-mail di alcuni dipendenti. Viene mostrata una finestra che comunica i dati mancanti.
		\end{itemize}
\end{itemize}
%------------------------------------------------
\paragraph{Estensioni}
%------------------------------------------------
\begin{itemize}
	\item \underline{Generazione nuova turnazione}
		\begin{itemize}
			\item Al passo (6) l'impiegato dell'ufficio amministrazione clicca su "Genera nuova turnazione". Si riparte quindi dal passo (5.a).
		\end{itemize}
\end{itemize}
%------------------------------------------------
\subsection{UC12: Gestione dei dati provenienti dalla timbratrice}
%------------------------------------------------
\paragraph{Descrizione}
%------------------------------------------------
Si vogliono estrarre i dati contenuti nei file XML, selezionando solamente i dati di interesse. Il gestore del personale può fare delle modifiche o eliminare alcuni dati. \\
Successivamente, confermando i dati, gli orari delle timbrature vengono aggregati per ogni dipendente, in modo da avere tutti gli orari di ingresso ed uscita del giorno.
%------------------------------------------------
\paragraph{Requisiti coperti}
%------------------------------------------------
FM00-FM05, SM09, SM10, SM20-SM23
%------------------------------------------------
\paragraph{Attori coinvolti}
%------------------------------------------------
Impiegato dell'ufficio amministrazione
%------------------------------------------------
\paragraph{Precondizioni}
%------------------------------------------------
E' disponibile il file \verb|.XML| contenente i dati delle timbrature. Questo file contiene i dati "grezzi", quindi non ancora elaborati.
%------------------------------------------------
\paragraph{Postcondizioni}
%------------------------------------------------
\begin{itemize}
	\item Le informazioni di interesse (orario, ID badge, tipo timbratura) vengono estratte e memorizzate all'interno del database.
	\item Nel caso si siano modificati alcuni dati, nel database sono presenti i dati aggiornati.
	\item Nel caso si siano eliminati alcuni dati, nel database non sono più presenti tali dati.
\end{itemize}
%------------------------------------------------
\paragraph{Processo}
%------------------------------------------------
Di seguito è descritto il processo:
\begin{enumerate}
	\item L'impiegato dell'ufficio amministrazione clicca sul pulsante "Importa file \verb|.XML|".
		\begin{enumerate}
			\item Viene mostrata una maschera per la selezione del file da importare.
		\end{enumerate}	
	\item L'impiegato dell'ufficio amministrazione seleziona il file che vuole importare. 
	\item Il sistema estrae i vari record delle timbrature tramite parsing, scorrendo i vari tag.
		\begin{enumerate}
			\item Viene mostrata una form contenente i dati importati, per un'eventuale modifica.
		\end{enumerate}		
	\item L'impiegato dell'ufficio amministrazione può fare una delle seguenti scelte:
		\begin{enumerate}
			\item \underline{Modifica dati}:
				\begin{enumerate}
						\item L'impiegato dell'ufficio amministrazione seleziona una timbratura che vuole modificare.
						\item L'impiegato dell'ufficio amministrazione effettua un doppio click sul dato che vuole modificare.
							\begin{itemize}
								\item I campi con i dati della timbratura diventano modificabili.
							\end{itemize}
						\item L'impiegato dell'ufficio amministrazione inserisce i dati corretti.
						\item L'impiegato dell'ufficio amministrazione preme il pulsante \verb|INVIO|.
							\begin{itemize}
								\item Viene aggiornato il contenuto della form di revisione dei dati importati.
								\item Si ritorna al passo (4).
							\end{itemize}
				\end{enumerate}
			\item \underline{Eliminazione dei dati}:
				\begin{enumerate}
						\item L'impiegato dell'ufficio amministrazione seleziona una timbratura che vuole eliminare.
						\item L'impiegato dell'ufficio amministrazione svuota tutti i campi della riga che vuole eliminate.
							\begin{itemize}
								\item Viene aggiornato il contenuto della form di revisione dei dati importati.
								\item Si ritorna al passo (4).
							\end{itemize}
				\end{enumerate}
		\end{enumerate}
	\item L'impiegato dell'ufficio amministrazione conferma i dati, tramite il pulsante "Salva".
		\begin{enumerate}
			\item Il sistema salva i dati grezzi, provenienti dalla timbratrice, con le eventuali modifiche effettuate manualmente dall'impiegato dell'ufficio amministrazione.
			\item Il sistema aggrega i dati di ogni dipendente, grazie al codice identificativo, al fine di avere per ogni dipendente tutti gli orari di ingresso/uscita giornalieri.
			\item I dati aggregati vengono inseriti all'interno del database.
		\end{enumerate}	
\end{enumerate}
%------------------------------------------------
\paragraph{Alternative}
%------------------------------------------------
\begin{itemize}
	\item \underline{Struttura file .XML errata}
		\begin{itemize}
			\item Al passo (2) l'impiegato dell'ufficio amministrazione seleziona un file che non soddisfa la struttura prevista (ad esempio con tag mancante o tag aggiuntivo). Il processo di parsing viene quindi interrotto e viene visualizzato un messaggio di errore.
		\end{itemize}
	\item \underline{Nessun file .XML selezionato}
		\begin{itemize}
			\item Al passo (2) l'impiegato dell'ufficio amministrazione non seleziona alcun file, oppure seleziona un file non \verb|.XML|. Il processo di parsing viene quindi interrotto e viene visualizzato un messaggio di errore.
		\end{itemize}
	\item \underline{Inserimento di una data errata}
		\begin{itemize}
			\item Al passo (4.a.iii) l'impiegato dell'ufficio amministrazione inserisce un valore errato per il campo "Data". Viene quindi mostrato un messaggio di errore e viene data la possibilità di modificare il campo errato.
		\end{itemize}
	\item \underline{Inserimento di un orario errato}
		\begin{itemize}
			\item Al passo (4.a.iii) l'impiegato dell'ufficio amministrazione inserisce un valore errato per il campo "Orario Timbratura". Viene quindi mostrato un messaggio di errore e viene data la possibilità di modificare il campo errato.
		\end{itemize}
\end{itemize}

%------------------------------------------------
\subsection{UC13: Gestione straordinari}
%------------------------------------------------
\paragraph{Descrizione}
%------------------------------------------------
Si vogliono inserire/aggiornare o eliminare i dati relativi alle ore straordinarie necessarie in un determinato periodo.
%------------------------------------------------
\paragraph{Requisiti coperti}
%------------------------------------------------
SM17-SM19, SM26
%------------------------------------------------
\paragraph{Attori coinvolti}
%------------------------------------------------
Impiegato dell'ufficio amministrazione
%------------------------------------------------
\paragraph{Precondizioni}
%------------------------------------------------
E' valida una di queste precondizioni:
\begin{itemize}
	\item Non è ancora stato inserito alcun dato riguardante le ore di straordinario necessarie.
	\item I dati sulle ore di straordinario necessarie sono già stati inseriti e devono essere aggiornati.
	\item I dati sulle ore di straordinario necessarie sono già stati inseriti e devono essere eliminati.
\end{itemize}
%------------------------------------------------
\paragraph{Postcondizioni}
%------------------------------------------------
In base all'azione eseguita è valida almeno una delle seguenti postcondizioni:
\begin{itemize}
	\item I nuovi dati riguardanti le ore di straordinario sono inseriti all'interno del database.
	\item I dati riguardanti le ore di straordinario già presenti nel database sono stati aggiornati.
	\item I dati riguardanti le ore di straordinario già presenti nel database sono stati eliminati.
\end{itemize}
%------------------------------------------------
\paragraph{Processo}
%------------------------------------------------
Di seguito è descritto il processo:
\begin{enumerate}
	\item L'impiegato dell'ufficio amministrazione clicca sul pulsante "Gestisci fabbisogno straordinari".
	\item L'impiegato dell'ufficio amministrazione ha a disposizione tre scelte:
		\begin{enumerate}
			\item Può cliccare sul pulsante "Inserisci".
			\item Può cliccare sul pulsante "Modifica".
			\item Può cliccare sul pulsante "Elimina".
		\end{enumerate}	
	\item Se l'impiegato dell'ufficio amministrazione ha cliccato su "Inserisci":
		\begin{enumerate}
			\item L'impiegato dell'ufficio amministrazione inserisce la data per la quale vuole inserire le ore di straordinario necessarie.
			\item L'impiegato dell'ufficio amministrazione inserisce l'attività per la quale sono necessarie le ore di straordinario.
			\item L'impiegato dell'ufficio amministrazione inserisce il numero di ore necessarie.
			\item L'impiegato dell'ufficio amministrazione preme il pulsante "Conferma" per confermare l'inserimento.
				\begin{enumerate}
					\item I dati vengono inseriti nel database.
				\end{enumerate}
		\end{enumerate}
	\item Se l'impiegato dell'ufficio amministrazione ha cliccato su "Modifica":
		\begin{enumerate}
			\item L'impiegato dell'ufficio amministrazione inserisce la data per la quale vuole modificare le ore di straordinario necessarie.
			\item L'impiegato dell'ufficio amministrazione inserisce l'attività per la quale vuole modificare le ore di straordinario necessarie.
				\begin{enumerate}
					\item Il campo contenente le ore di straordinario viene pre-compilato, indicando i dati già inseriti nel database.
				\end{enumerate}
			\item L'impiegato dell'ufficio amministrazione inserisce il nuovo numero di ore necessarie.
			\item L'impiegato dell'ufficio amministrazione preme il pulsante "Conferma" per confermare l'aggiornamento.
				\begin{enumerate}
					\item I dati vengono aggiornati nel database.
				\end{enumerate}
		\end{enumerate}
	\item Se l'impiegato dell'ufficio amministrazione ha cliccato su "Elimina":
		\begin{enumerate}
			\item L'impiegato dell'ufficio amministrazione inserisce la data per la quale vuole eliminare le ore di straordinario necessarie.
			\item L'impiegato dell'ufficio amministrazione inserisce l'attività per la quale vuole eliminare le ore di straordinario necessarie.
			\item L'impiegato dell'ufficio amministrazione preme il pulsante "Conferma eliminazione" per confermare la cancellazione.
				\begin{enumerate}
					\item I dati vengono eliminati dal database.
				\end{enumerate}
		\end{enumerate}	
\end{enumerate}
%------------------------------------------------
\paragraph{Alternative}
%------------------------------------------------
\begin{itemize}
	\item \underline{Inserimento dati errati}
		\begin{itemize}
			\item Ai passi (3.*), (4.*) o (5.*) vengono inseriti dei dati errati. L'operazione viene annullata e viene mostrato un messaggio di errore.
			\item Per dati errati si intende:
				\begin{itemize}
					\item Date non corrette.
					\item Attività non esistenti.
					\item Numero di ore non corretto (\textit{ad esempio $<0$}).
					\item Qualsiasi dato che non rispetta vincoli di dominio (\textit{ad esempio superamento del massimale di ore di straordinario possibili in una giornata}).
				\end{itemize}
		\end{itemize}
	\item \underline{Modifica di una tupla non presente}
		\begin{itemize}
			\item Al passo (4.b) l'impiegato dell'ufficio amministrazione tenta di modificare una tupla non presente. Viene mostrato un messaggio di errore e viene data la possibilità di modificare i dati inseriti.
		\end{itemize}
	\item \underline{Eliminazione di una tupla non presente}
		\begin{itemize}
			\item Al passo (5.c) l'impiegato dell'ufficio amministrazione tenta di eliminare una tupla non presente. Viene mostrato un messaggio di errore e viene data la possibilità di modificare i dati inseriti.
		\end{itemize}
\end{itemize}

%------------------------------------------------
\subsection{UC14: Gestione giorni di malattia/ferie}
%------------------------------------------------
\paragraph{Descrizione}
%------------------------------------------------
Si vogliono inserire/aggiornare o eliminare i dati relativi ai giorni di malattia o di ferie comunicati dai dipendenti.
%------------------------------------------------
\paragraph{Requisiti coperti}
%------------------------------------------------
SM09, SM14-SM16
%------------------------------------------------
\paragraph{Attori coinvolti}
%------------------------------------------------
Impiegato dell'ufficio amministrazione
%------------------------------------------------
\paragraph{Precondizioni}
%------------------------------------------------
E' valida una di queste precondizioni:
\begin{itemize}
	\item Non è ancora stato inserito alcun dato riguardante i giorni di malattia o di ferie per un dipendente.
	\item I dati sui giorni di malattia o ferie di un dipendente sono già stati inseriti e devono essere aggiornati.
	\item I dati sui giorni di malattia o ferie di un dipendente sono già stati inseriti e devono essere eliminati.
\end{itemize}
%------------------------------------------------
\paragraph{Postcondizioni}
%------------------------------------------------
In base all'azione eseguita è valida almeno una delle seguenti postcondizioni:
\begin{itemize}
	\item I nuovi dati riguardanti i giorni di malattia o ferie sono inseriti all'interno del database.
	\item I dati riguardanti i giorni di malattia o ferie già presenti nel database sono stati aggiornati.
	\item I dati riguardanti i giorni di malattia o ferie già presenti nel database sono stati eliminati.
\end{itemize}
%------------------------------------------------
\paragraph{Processo}
%------------------------------------------------
Di seguito è descritto il processo:
\begin{enumerate}
	\item L'impiegato dell'ufficio amministrazione clicca sul pulsante "Gestisci giorni di ferie"/"Gestisci giorni di malattia".
	\item L'impiegato dell'ufficio amministrazione ha a disposizione tre scelte:
		\begin{enumerate}
			\item Può cliccare sul pulsante "Inserisci".
			\item Può cliccare sul pulsante "Modifica".
			\item Può cliccare sul pulsante "Elimina".
		\end{enumerate}	
	\item Se l'impiegato dell'ufficio amministrazione ha cliccato su "Inserisci":
		\begin{enumerate}
			\item L'impiegato dell'ufficio amministrazione inserisce l'identificativo del dipendente in ferie/malattia.
			\item L'impiegato dell'ufficio amministrazione inserisce l'intervallo di date per il quale il dipendente è in ferie/malattia.
			\item L'impiegato dell'ufficio amministrazione preme il pulsante "Conferma" per confermare l'inserimento.
				\begin{enumerate}
					\item I dati vengono inseriti nel database.
				\end{enumerate}
		\end{enumerate}
	\item Se l'impiegato dell'ufficio amministrazione ha cliccato su "Modifica":
		\begin{enumerate}
			\item L'impiegato dell'ufficio amministrazione inserisce l'identificativo del dipendente in ferie/malattia.
			\item L'impiegato dell'ufficio amministrazione imposta l'intervallo di date contenente i giorni da modificare.
				\begin{enumerate}
					\item Vengono mostrati i dati del dipendente per il mese selezionato.
				\end{enumerate}
			\item L'impiegato dell'ufficio amministrazione modifica lo stato di uno o più giorni (ferie-no ferie / malattia-no malattia) e clicca sul pulsante "Conferma".
				\begin{enumerate}
					\item I dati vengono aggiornati nel database.
				\end{enumerate}
		\end{enumerate}
	\item Se l'impiegato dell'ufficio amministrazione ha cliccato su "Elimina":
		\begin{enumerate}
			\item L'impiegato dell'ufficio amministrazione inserisce l'identificativo del dipendente in ferie/malattia.
			\item L'impiegato dell'ufficio amministrazione imposta l'intervallo di date contenente i giorni da eliminare.
				\begin{enumerate}
					\item Vengono mostrati i dati del dipendente per il mese selezionato.
				\end{enumerate}
			\item L'impiegato dell'ufficio amministrazione elimina uno o più giorni e clicca sul pulsante "Conferma".
				\begin{enumerate}
					\item I dati vengono eliminati dal database.
				\end{enumerate}
		\end{enumerate}	
\end{enumerate}
%------------------------------------------------
\paragraph{Alternative}
%------------------------------------------------
\begin{itemize}
	\item \underline{Inserimento dati errati}
		\begin{itemize}
			\item Ai passi (3.*), (4.*) o (5.*) vengono inseriti dei dati errati. L'operazione viene annullata e viene mostrato un messaggio di errore.
			\item Per dati errati si intende:
				\begin{itemize}
					\item Date non corrette.
					\item Identificativo dipendente non esistente.
					\item Qualsiasi dato che non rispetta vincoli di dominio.
				\end{itemize}
		\end{itemize}
	\item \underline{Inserimento di dati incompatibili}
		\begin{itemize}
			\item Ai passi (3.*), (4.*) o (5.*) vengono inseriti dei dati incompatibili. L'operazione viene annullata e viene mostrato un messaggio di errore.
			\item Per dati incompatibili si intende il caso in cui vengano inserite una o più giornate di ferie/malattia per un dipendente che ha già superato il monte ore disponibile.
		\end{itemize}
	\item \underline{Aggiornamento di una tupla non presente}
		\begin{itemize}
			\item Al passo (4.c) l'impiegato dell'ufficio amministrazione tenta di aggiornare una tupla non presente. Viene mostrato un messaggio di errore e viene data la possibilità di modificare i dati inseriti.
		\end{itemize}
	\item \underline{Eliminazione di una tupla non presente}
		\begin{itemize}
			\item Al passo (5.c) l'impiegato dell'ufficio amministrazione tenta di eliminare una tupla non presente. Viene mostrato un messaggio di errore e viene data la possibilità di modificare i dati inseriti.
		\end{itemize}
\end{itemize}

%------------------------------------------------
\subsection{UC15: Visualizzazione e calcolo giornaliero ore di lavoro}
%------------------------------------------------
\paragraph{Descrizione}
%------------------------------------------------
Si vogliono calcolare le ore di lavoro relative ai dipendenti, a partire dai dati derivanti dal parsing del file \verb|.XML| (si veda UC12).
%------------------------------------------------
\paragraph{Requisiti coperti}
%------------------------------------------------
SM00-SM02, SM04, SM25, SM27
%------------------------------------------------
\paragraph{Attori coinvolti}
%------------------------------------------------
Impiegato dell'ufficio amministrazione
%------------------------------------------------
\paragraph{Precondizioni}
%------------------------------------------------
\begin{itemize}
	\item I dati relativi agli orari giornalieri dei dipendenti non sono presenti.
	\item I dati grezzi (con eventuale modifica manuale o revisione) sono già stati importati dal file \verb|.XML| (si veda UC12).
\end{itemize}
%------------------------------------------------
\paragraph{Postcondizioni}
%------------------------------------------------
Per ogni dipendente sono calcolate le ore giornaliere di lavoro ordinario, straordinario, di ferie, di permesso e malattia.
%------------------------------------------------
\paragraph{Processo}
%------------------------------------------------
Di seguito è descritto il processo:
\begin{enumerate}
	\item L'impiegato dell'ufficio amministrazione clicca sul pulsante "Visualizza orari dipendenti".
	\item L'impiegato dell'ufficio amministrazione clicca sul pulsante "Visualizza orari giornalieri".
		\begin{enumerate}
			\item Viene richiesto all'impiegato dell'ufficio amministrazione di inserire la data di interesse.
		\end{enumerate}
	\item L'impiegato dell'ufficio amministrazione inserisce la data di interesse e preme sul pulsante "Conferma" per avviare il processo.
		\begin{enumerate}
			\item Viene avviato il processo di calcolo degli orari del giorno selezionato:
				\begin{itemize}
					\item \underline{Lavoro ordinario}: se gli orari rientrano nelle fasce di lavoro ordinario, in base al turno assegnato a ciascun dipendente.
					\item \underline{Lavoro straordinario}: se un dipendente supera gli orari di lavoro ordinario.
					\item \underline{Ferie}: se il giorno d'interesse è stato inserito, per il dipendente considerato, come giorno di ferie (si veda UC14).
					\item \underline{Permesso}: se le ore di ordinario sono inferiori a quelle giornaliere contrattuali.
					\item \underline{Malattia}: se il giorno d'interesse è stato inserito, per il dipendente considerato, come giorno di malattia (si veda UC14).
				\end{itemize}
			\item Viene mostrata una maschera riassuntiva in cui, per ogni dipendente, è indicato il numero di ore giornaliere di lavoro ordinario, straordinario, di ferie, di permesso e di malattia.
		\end{enumerate}	
\end{enumerate}
%------------------------------------------------
\paragraph{Alternative}
%------------------------------------------------
\begin{itemize}
	\item \underline{Dipendente in ferie presente}
		\begin{itemize}
			\item Al passo (3), il sistema rileva una timbratura per un dipendente che è considerato in ferie, in base ai dati inseriti. Viene, quindi, mostrato un messaggio di errore informativo.
		\end{itemize}
	\item \underline{Dipendente in malattia presente}
		\begin{itemize}
			\item Al passo (3), il sistema rileva una timbratura per un dipendente che è considerato in malattia, in base ai dati inseriti. Viene, quindi, mostrato un messaggio di errore informativo.
		\end{itemize}
\end{itemize}
%------------------------------------------------
\paragraph{Estensioni}
%------------------------------------------------
\begin{itemize}
	\item \underline{Modifica orari timbratura}
		\begin{itemize}
			\item Al passo (3.b), l'impiegato dell'ufficio amministrazione può modificare gli orari di ingresso e uscita del dipendente.
			\item In questo caso, una volta confermata la modifica, viene ripetuto il calcolo delle ore.
		\end{itemize}
\end{itemize}
%------------------------------------------------
\subsection{UC16: Visualizzazione e calcolo mensile ore di lavoro}
%------------------------------------------------
\paragraph{Descrizione}
%------------------------------------------------
Si vogliono calcolare le ore di lavoro relative ai dipendenti, a partire dai dati derivanti dal parsing del file \verb|.XML| (si veda UC12).
%------------------------------------------------
\paragraph{Requisiti coperti}
%------------------------------------------------
SM05-SM08, SM13, SM24, SM25
%------------------------------------------------
\paragraph{Attori coinvolti}
%------------------------------------------------
Impiegato dell'ufficio amministrazione
%------------------------------------------------
\paragraph{Precondizioni}
%------------------------------------------------
\begin{itemize}
	\item I dati relativi agli orari mensili dei dipendenti non sono presenti.
	\item I dati grezzi (con eventuale modifica manuale o revisione) sono già stati importati dal file \verb|.XML| (si veda UC12).
\end{itemize}
%------------------------------------------------
\paragraph{Postcondizioni}
%------------------------------------------------
Per ogni dipendente sono calcolate le ore mensili di lavoro ordinario, straordinario, di ferie, di permesso e malattia.
%------------------------------------------------
\paragraph{Processo}
%------------------------------------------------
Di seguito è descritto il processo:
\begin{enumerate}
	\item L'impiegato dell'ufficio amministrazione clicca sul pulsante "Visualizza orari dipendenti".
	\item L'impiegato dell'ufficio amministrazione clicca sul pulsante "Visualizza orari mensili".
		\begin{enumerate}
			\item Viene richiesto all'impiegato dell'ufficio amministrazione di inserire il mese e l'anno di interesse.
			\item Viene richiesto all'impiegato dell'ufficio amministrazione di inserire l'identificativo del dipendente per il quale vuole conoscere gli orari mensili.
		\end{enumerate}
	\item L'impiegato dell'ufficio amministrazione inserisce l'anno ed il mese di interesse.
	\item L'impiegato dell'ufficio amministrazione inserisce l'identificativo del dipendente per il quale vuole conoscere gli orari mensili e preme sul pulsante "Conferma" per avviare il processo.
		\begin{enumerate}
			\item Viene avviato il processo di calcolo degli orari del mese selezionato:
				\begin{itemize}
					\item \underline{Lavoro ordinario}: se gli orari rientrano nelle fasce di lavoro ordinario, in base al turno assegnato ogni giorno al dipendente.
					\item \underline{Lavoro straordinario}: se il dipendente supera gli orari di lavoro ordinario.
					\item \underline{Ferie}: se un determinato giorno è stato inserito, per il dipendente considerato, come giorno di ferie (si veda UC14).
					\item \underline{Permesso}: se le ore di ordinario sono inferiori a quelle giornaliere contrattuali.
					\item \underline{Malattia}: se un determinato giorno è stato inserito, per il dipendente considerato, come giorno di malattia (si veda UC14).
				\end{itemize}
			\item Viene mostrata una maschera riassuntiva in cui, per il dipendente desiderato, è indicato il numero di ore mensili di lavoro ordinario, straordinario, di ferie, di permesso e di malattia.
		\end{enumerate}	
\end{enumerate}
%------------------------------------------------
\paragraph{Alternative}
%------------------------------------------------
\begin{itemize}
	\item \underline{Dipendente in ferie presente}
		\begin{itemize}
			\item Al passo (4), il sistema rileva una timbratura per il dipendente in un giorno che è stato precedentemente inserito come giorno di ferie. Viene, quindi, mostrato un messaggio di errore informativo.
		\end{itemize}
	\item \underline{Dipendente in malattia presente}
		\begin{itemize}
			\item Al passo (4), il sistema rileva una timbratura per il dipendente in un giorno che è stato precedentemente inserito come giorno di malattia. Viene, quindi, mostrato un messaggio di errore informativo.
		\end{itemize}
	\item \underline{Superamento limite di lavoro straordinario}
		\begin{itemize}
			\item Al passo (4), il sistema rileva una quantità di ore di lavoro straordinario superiore al limite. In questo caso viene mostrato un messaggio di errore.
		\end{itemize}
\end{itemize}
%------------------------------------------------
\paragraph{Estensioni}
%------------------------------------------------
\begin{itemize}
	\item \underline{Stampa report mensile dipendente}
		\begin{itemize}
			\item Al passo (4), l'impiegato dell'ufficio amministrazione può cliccare sul pulsante "Stampa" per esportare un PDF contenente il report mensile degli orari del dipendente.
			\item Si procede come da UC19.
		\end{itemize}
	\item \underline{Modifica orari timbratura}
		\begin{itemize}
			\item Al passo (3.b), l'impiegato dell'ufficio amministrazione può modificare gli orari di ingresso e uscita del dipendente.
			\item In questo caso, una volta confermata la modifica, viene ripetuto il calcolo delle ore.
		\end{itemize}
\end{itemize}
%------------------------------------------------
\subsection{UC17: Applicazione incremento percentuale straordinari}
%------------------------------------------------
\paragraph{Descrizione}
%------------------------------------------------
Si vuole applicare un incremento percentuale in caso di lavoro straordinario o di lavoro notturno (prima delle 08:00 e dopo le 22:00).
%------------------------------------------------
\paragraph{Requisiti coperti}
%------------------------------------------------
SM11
%------------------------------------------------
\paragraph{Attori coinvolti}
%------------------------------------------------
Gestione degli orari dei dipendenti
%------------------------------------------------
\paragraph{Precondizioni}
%------------------------------------------------
Ad ogni orario del dipendente è applicata la paga base oraria.
%------------------------------------------------
\paragraph{Postcondizioni}
%------------------------------------------------
Viene applicato l'incremento percentuale sia alle ore di lavoro straordinario che in quelle di lavoro ordinario.
%------------------------------------------------
\paragraph{Processo}
%------------------------------------------------
Di seguito è descritto il processo:
\begin{enumerate}
	\item Il sistema seleziona dal database le ore di lavoro straordinario o notturno.
	\item Il sistema applica l'incremento percentuale alle ore notturne.
	\item Il sistema applica l'incremento percentuale alle ore di straordinario.
	\item I dati vengono memorizzati all'interno del database.
\end{enumerate}


%------------------------------------------------
\subsection{UC18: Calcolo stipendio}
%------------------------------------------------
\paragraph{Descrizione}
%------------------------------------------------
Si vuole calcolare lo stipendio mensile di ogni dipendente.
%------------------------------------------------
\paragraph{Requisiti coperti}
%------------------------------------------------
SM12
%------------------------------------------------
\paragraph{Attori coinvolti}
%------------------------------------------------
Impiegato dell'ufficio amministrazione
%------------------------------------------------
\paragraph{Precondizioni}
%------------------------------------------------
Sono caricate le timbrature di ogni dipendente.
%------------------------------------------------
\paragraph{Postcondizioni}
%------------------------------------------------
Per ogni dipendente è calcolato lo stipendio mensile.
%------------------------------------------------
\paragraph{Processo}
%------------------------------------------------
Di seguito è descritto il processo:
\begin{enumerate}
	\item L'impiegato dell'ufficio amministrazione clicca sul pulsante "Calcola stipendi".
		\begin{enumerate}
			\item Viene mostrata una form con la possibilità di selezionare il mese e l'anno di interesse.
		\end{enumerate}
	\item L'impiegato dell'ufficio amministrazione seleziona l'anno ed il mese di interesse.
	\item L'impiegato dell'ufficio amministrazione clicca sul pulsante "Calcola".
		\begin{enumerate}
			\item Il sistema calcola in automatico gli stipendi per tutti i dipendenti.
			\item Nel caso di straordinari o orari notturni, viene applicato un incremento percentuale come da UC17.
			\item Il sistema applica eventuali tassazioni.
			\item Viene mostrata una form con l'indicazione, per ogni dipendente, del relativo stipendio.
		\end{enumerate}
\end{enumerate}
%------------------------------------------------
\paragraph{Alternative}
%------------------------------------------------
\begin{itemize}
	\item \underline{Inserimento anno o mese errato}
		\begin{itemize}
			\item Al passo (2), l'impiegato dell'ufficio amministrazione inserisce dei valori errati per l'anno o il mese. Viene, quindi, mostrato un messaggio di errore e viene data la possibilità di correggere i valori inseriti.
		\end{itemize}
	\item \underline{Inserimento anno o mese futuro}
		\begin{itemize}
			\item Al passo (2), l'impiegato dell'ufficio amministrazione inserisce dei valori futuri per l'anno o il mese (\textit{quindi per i quali ancora non si hanno indicazioni sulle timbrature}). Viene, quindi, mostrato un messaggio di errore e viene data la possibilità di correggere i valori inseriti.
		\end{itemize}
\end{itemize}

%------------------------------------------------
\subsection{UC19: Generazione report dipendente}
%------------------------------------------------
\paragraph{Descrizione}
%------------------------------------------------
Si vuole stampare un report, per un dipendente selezionato, che rispecchia la situazione mensile lavorativa.
%------------------------------------------------
\paragraph{Requisiti coperti}
%------------------------------------------------
SM03, SM13
%------------------------------------------------
\paragraph{Attori coinvolti}
%------------------------------------------------
Impiegato dell'ufficio amministrazione
%------------------------------------------------
\paragraph{Precondizioni}
%------------------------------------------------
Sono già stati caricati i dati di ore di ordinario, straordinario, ferie, permessi e malattia per il dipendente.
%------------------------------------------------
\paragraph{Postcondizioni}
%------------------------------------------------
Viene prodotto un report mensile del dipendente, con statistiche riguardanti le varie tipologie di lavoro svolto, resoconto delle timbrature e stipendio.
%------------------------------------------------
\paragraph{Processo}
%------------------------------------------------
Di seguito è descritto il processo:
\begin{enumerate}
	\item L'impiegato dell'ufficio amministrazione clicca sul pulsante "Stampa report dipendente".
		\begin{enumerate}
			\item Viene mostrata una form con la possibilità di selezionare il mese e l'anno di interesse, oltre che l'identificativo del dipendente per il quale si vuole ottenere il report.
		\end{enumerate}
	\item L'impiegato dell'ufficio amministrazione seleziona l'anno ed il mese di interesse.
	\item L'impiegato dell'ufficio amministrazione seleziona l'identificativo del dipendendente di interesse.
	\item L'impiegato dell'ufficio amministrazione clicca sul pulsante "Genera".
		\begin{enumerate}
			\item Il sistema aggrega tutte le informazioni necessarie e produce il report PDF per il dipendente selezionato.
		\end{enumerate}
\end{enumerate}
%------------------------------------------------
\paragraph{Alternative}
%------------------------------------------------
\begin{itemize}
	\item \underline{Inserimento anno o mese errato}
		\begin{itemize}
			\item Al passo (2), l'impiegato dell'ufficio amministrazione inserisce dei valori errati per l'anno o il mese. Viene, quindi, mostrato un messaggio di errore e viene data la possibilità di correggere i valori inseriti.
		\end{itemize}
	\item \underline{Inserimento anno o mese futuro}
		\begin{itemize}
			\item Al passo (2), l'impiegato dell'ufficio amministrazione inserisce dei valori futuri per l'anno o il mese (\textit{quindi per i quali ancora non si hanno indicazioni sulle timbrature}). Viene, quindi, mostrato un messaggio di errore e viene data la possibilità di correggere i valori inseriti.
		\end{itemize}
	\item \underline{Inserimento identificativo dipendente errato}
		\begin{itemize}
			\item Al passo (3), l'impiegato dell'ufficio amministrazione inserisce un identificativo di un dipendente non esistente. Viene, quindi, mostrato un messaggio di errore e viene data la possibilità di correggere i valori inseriti.
		\end{itemize}
\end{itemize}

%------------------------------------------------
\subsection{UC20: Prenotazione per straordinari}
%------------------------------------------------
\paragraph{Descrizione}
%------------------------------------------------
Si vuole fare in modo che un dipendente possa prenotarsi per un certo numero di ore di straordinario previste.
%------------------------------------------------
\paragraph{Requisiti coperti}
%------------------------------------------------
SM28
%------------------------------------------------
\paragraph{Attori coinvolti}
%------------------------------------------------
Dipendente
%------------------------------------------------
\paragraph{Precondizioni}
%------------------------------------------------
Sono già stati caricati i dati relativi alle ore di straordinario necessarie, previste dall'azienda.
%------------------------------------------------
\paragraph{Postcondizioni}
%------------------------------------------------
Nel database è registrata la prenotazione del dipendente per un certo numero di ore di straordinario previste.
%------------------------------------------------
\paragraph{Processo}
%------------------------------------------------
Di seguito è descritto il processo:
\begin{enumerate}
	\item Il dipendente clicca su "Prenotazione straordinari".
		\begin{enumerate}
			\item Viene mostrata una form con l'elenco degli straordinari previsti (si veda UC13), per i quali non sono già terminate le prenotazioni.\\
				Si dà inoltre la possibilità di filtrare le date, inserendo l'intervallo di date desiderato.
		\end{enumerate}
	\item Il dipendente seleziona la riga relativa alla data di interesse.
	\item Il dipendente clicca su "Prenota".
	\item Il dipendente inserisce il numero di ore per cui si prenota.
		\begin{enumerate}
			\item Il dato inserito viene memorizzato nel database.
			\item Viene mostrato al dipendente un messaggio di conferma.
		\end{enumerate}
\end{enumerate}
%------------------------------------------------
\paragraph{Alternative}
%------------------------------------------------
\begin{itemize}
	\item \underline{Inserimento date errate}
		\begin{itemize}
			\item Al passo (2.a), il dipendente inserisce, per filtrare i risultati, delle date errate. Viene, quindi, mostrato un messaggio di errore e viene data la possibilità di correggere i valori inseriti.
		\end{itemize}
	\item \underline{Inserimento ore sopra il limite}
		\begin{itemize}
			\item Al passo (4), il dipendente inserisce un numero di ore superiori a quelle disponibili. Viene, quindi, mostrato un messaggio di errore e viene data la possibilità di correggere i valori inseriti.
			\item Lo stesso comportamento si ha quando il dipendente si prenota, nella stessa settimana o nello stesso mese, per un numero di ore che eccede quelle permesse legalmente.
		\end{itemize}
	\item \underline{Prenotazione per date passate}
		\begin{itemize}
			\item Al passo (3), il dipende cerca di prenotarsi per una data passata. Viene, quindi, mostrato un messaggio di errore.
		\end{itemize}
\end{itemize}
%------------------------------------------------
\paragraph{Estensioni}
%------------------------------------------------
\begin{itemize}
	\item \underline{Selezione date per filtro}
		\begin{itemize}
			\item Al passo (2.a) il dipendente può inserire una o due date, per filtrare quelle memorizzare.
			\item In questo caso vengono mostrate solamente le date con straordinari disponibili, comprese all'interno dell'intervallo.
		\end{itemize}
\end{itemize}
%------------------------------------------------
\subsection{UC21: Verifica copertura turni assegnati}
%------------------------------------------------
\paragraph{Descrizione}
%------------------------------------------------
Si vuole verificare che un dipendente abbia coperto tutti i turni che gli sono stati assegnati.
%------------------------------------------------
\paragraph{Requisiti coperti}
%------------------------------------------------
SM25
%------------------------------------------------
\paragraph{Attori coinvolti}
%------------------------------------------------
Impiegato ufficio amministrazione
%------------------------------------------------
\paragraph{Precondizioni}
%------------------------------------------------
Sono già stati caricati i turni per le date che si vogliono analizzare. Inoltre deve già essere caricato anche il dipendente all'interno del database.
%------------------------------------------------
\paragraph{Postcondizioni}
%------------------------------------------------
Viene mostrata una finestra riepilogativa, indicante quali turni sono stati effettivamente coperti dal dipendente, e quali no.
%------------------------------------------------
\paragraph{Processo}
%------------------------------------------------
Di seguito è descritto il processo:
\begin{enumerate}
	\item L'impiegato dell'ufficio amministrazione clicca su "Verifica copertura turni".
		\begin{enumerate}
			\item Viene mostrata una form nella quale l'impiegato dell'ufficio amministrazione può selezionare il dipendente e l'intervallo di interesse.
		\end{enumerate}
	\item L'impiegato dell'ufficio amministrazione inserisce il dipendente per il quale vuole avviare il processo di verifica.
	\item L'impiegato dell'ufficio amministrazione inserisce l'intervallo per il quale vuole avviare il processo di verifica.
	\item L'impiegato dell'ufficio amministrazione clicca su "Verifica".
		\begin{enumerate}
			\item Viene mostrata una form nella quale vengono mostrati i turni che sono stati effettivamente coperti dal dipendente.
		\end{enumerate}
\end{enumerate}
%------------------------------------------------
\paragraph{Alternative}
%------------------------------------------------
\begin{itemize}
	\item \underline{Inserimento date errate}
		\begin{itemize}
			\item Al passo (3), l'impiegato dell'ufficio amministrazione inserisce, per filtrare i risultati, delle date errate. Viene, quindi, mostrato un messaggio di errore e viene data la possibilità di correggere i valori inseriti.
		\end{itemize}
\end{itemize}
%------------------------------------------------
\section{Use case diagram}
%------------------------------------------------
\addfig{Specifiche/img/}{UseCase}{1}{Use case diagram}{Use case diagram}