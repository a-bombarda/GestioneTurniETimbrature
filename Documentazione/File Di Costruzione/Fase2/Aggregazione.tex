\chapter{Aggregazione degli orari}
%------------------------------------------------
\section{Introduzione}
Una volta che i dati grezzi, relativi alle singole timbrature, sono stati inseriti all'interno del database, il passo successivo è quello di aggregare i dati di ogni singolo giorno per ogni dipendente.\\
Di seguito verrà riportata l'analisi di questa funzionalità.
%------------------------------------------------
\section{Funzione di aggregazione}
Il compito della funzione di aggregazione può essere riassunto dai seguenti passi:
\begin{itemize}
	\item Individuare tutte le timbrature fatte da un determinato dipendente in una determinata data.
	\item Ordinare le timbrature individuate in ordine crescente.
	\item Creare un "orario" completo, comprendente tutte le entrate ed uscite giornaliere del dipendente.
	\item Per ogni giornata di lavoro calcolare:
		\begin{itemize}
			\item Ore di lavoro ordinario
			\item Ore di lavoro straordinario
			\item Ore di ferie/permesso
		\end{itemize}
\end{itemize}
La funzione può essere implementata dal seguente pseudocodice:
\lstset{
    language=java,
    tabsize=2,
    frame=lines,
    caption=Pseudocodice della funzione di aggregazione,
    label=code:pseudoAggregazione,
    frame=shadowbox,
    rulesepcolor=\color{gray},
    xleftmargin=20pt,
    framexleftmargin=15pt,
    keywordstyle=\color{blue}\bf,
    commentstyle=\color{Green},
    stringstyle=\color{red},
    numbers=left,
    numberstyle=\tiny,
    numbersep=5pt,
    breaklines=true,
    showstringspaces=false,
    basicstyle=\footnotesize,
    emph={fine,funzione,min,max},emphstyle={\color{magenta}}}

\lstinputlisting[language=java]{Fase2/src/aggregazione}
\noindent
Come da Codice \ref{code:pseudoAggregazione}, la funzione \verb|aggregaOrari| comprende al suo interno diverse chiamate a procedure che si occupano di svolgere i vari passi necessari al processo di aggregazione.
%------------------------------------------------
\subsection{Funzione di creazione delle righe vuote}
La funzione di creazione delle righe vuote si occupa di predisporre la tabella degli \verb|Orari|, completandola con tutte le giornate, per ogni dipendente, comprese tra la minima e la massima data contenuta all'interno del file \verb|XML| importato dalla timbratrice.\\

\noindent
La funzione può essere rappresentata dal \textit{flow-chart} in figura \ref{fig:CreazioneRigheVuote} ed implementata dallo pseudocodice riportato in Codice \ref{code:PseudoVuote}.

\addfig{Fase2/img/}{CreazioneRigheVuote}{0.9}{Flow-chart per la funzione di creazione delle righe vuote}{Flow-chart per la funzione di creazione delle righe vuote}

\lstset{
    language=java,
    tabsize=2,
    frame=lines,
    caption=Pseudocodice della funzione di creazione delle righe vuote,
    label=code:PseudoVuote,
    frame=shadowbox,
    rulesepcolor=\color{gray},
    xleftmargin=20pt,
    framexleftmargin=15pt,
    keywordstyle=\color{blue}\bf,
    commentstyle=\color{Green},
    stringstyle=\color{red},
    numbers=left,
    numberstyle=\tiny,
    numbersep=5pt,
    breaklines=true,
    showstringspaces=false,
    basicstyle=\footnotesize,
    emph={fine,funzione,min,max,for,each,end,between, in, and},emphstyle={\color{magenta}}}

\lstinputlisting[language=java]{Fase2/src/vuote}
%------------------------------------------------
\subsubsection{Analisi della complessità}
Consideriamo:
\begin{itemize}
	\item $n$, il numero di dipendenti dell'azienda
	\item $m$, il numero di giorni tra \verb|dataMin| e \verb|dataMax|
\end{itemize}
La complessità di questa funzione, considerando le operazioni di accesso al database di costo unitario $O(1)$ (\textit{semplificazione}), può essere considerata pari a
\begin{equation}
	O(n\cdot m)
\end{equation}
Inoltre, supponendo di lavorare con un committente che abbia un numero significativo di dipendenti e che importi i dati della timbratrice con un intervallo abbastanza breve, possiamo considerare
\begin{equation}
	n \gg m
\end{equation}
ovvero, passiamo ad una complessità della funzione pari a
\begin{equation}
	O(n)
\end{equation}
%------------------------------------------------
\subsection{Funzione di posizionamento degli orari}
La funzione di posizionamento degli orari si occupa di copiare, in ciascuna delle righe della tabella degli \verb|Orari| predisposte dalla funzione precedente, tutte le timbrature effettuate da un dipendente in una specifica giornata. In modo è possibile avere la situazione complessiva delle entrate e delle uscite del dipendente dall'area produttiva.\\
La copia delle timbrature deve essere effettuata in ordine orario crescente.\\

\noindent
La funzione può essere rappresentata dal \textit{flow-chart} in figura \ref{fig:Posizionamento} ed implementata dallo pseudocodice riportato in Codice \ref{code:PseudoPosizionamento}.

\addfig{Fase2/img/}{Posizionamento}{1}{Flow-chart per la funzione di posizionamento degli orari}{Flow-chart per la funzione di posizionamento degli orari}

\lstset{
    language=java,
    tabsize=2,
    frame=lines,
    caption=Pseudocodice della funzione di posizionamento degli orari,
    label=code:PseudoPosizionamento,
    frame=shadowbox,
    rulesepcolor=\color{gray},
    xleftmargin=20pt,
    framexleftmargin=15pt,
    keywordstyle=\color{blue}\bf,
    commentstyle=\color{Green},
    stringstyle=\color{red},
    numbers=left,
    numberstyle=\tiny,
    numbersep=5pt,
    breaklines=true,
    showstringspaces=false,
    basicstyle=\footnotesize,
    emph={fine,funzione,min,max,for,each,end,between, in, and},emphstyle={\color{magenta}}}

\lstinputlisting[language=java]{Fase2/src/Posizionamento}
%------------------------------------------------
\subsubsection{Analisi della complessità}
Consideriamo:
\begin{itemize}
	\item $n$, il numero di dipendenti dell'azienda
	\item $m$, il numero di giorni tra \verb|dataMin| e \verb|dataMax|
	\item Ordinamento eseguito tramite un algoritmo ottimo, con complessità $O(n\cdot \log{n})$.\\
		Questa assunzione vale solamente a livello teorico, in quanto l'ordinamento verrà eseguito tramite una query SQL e, di conseguenza, gestito dal DMBS con algoritmo non-noto.
\end{itemize}
La complessità di questa funzione, considerando le operazioni di accesso al database di costo unitario $O(1)$ (\textit{semplificazione}), può essere considerata pari a
\begin{equation}
	O(n\cdot m \cdot n \cdot \log{n}) = O(n^2 \cdot m \cdot \log{n})
\end{equation}
Inoltre, supponendo di lavorare con un committente che abbia un numero significativo di dipendenti e che importi i dati della timbratrice con un intervallo abbastanza breve, possiamo considerare
\begin{equation}
	n \gg m
\end{equation}
ovvero, passiamo ad una complessità della funzione pari a
\begin{equation}
	O(n^2 \cdot \log{n})
\end{equation}
%------------------------------------------------
\subsection{Funzione di elaborazione degli orari}
La funzione di elaborazione degli orari deve, basandosi su quanto prodotto dalla funzione precedente, calcolare, per ogni giornata e per ogni dipendente, le ore di:
\begin{itemize}
	\item Ordinario
	\item Straordinario
	\item Ferie/Permessi
\end{itemize}
Per scelta del committente, gli il numero di ore viene gestito a scatti di mezz'ora, arrotondando alla mezz'ora più vicina.\\

\noindent
La funzione può essere rappresentata dal \textit{flow-chart} in figura \ref{fig:Elaborazione} ed implementata dallo pseudocodice riportato in Codice \ref{code:PseudoElabora}.


\addfig{Fase2/img/}{Elaborazione}{1}{Flow-chart per la funzione di elaborazione degli orari}{Flow-chart per la funzione di elaborazione degli orari}

\lstset{
    language=java,
    tabsize=2,
    frame=lines,
    caption=Pseudocodice della funzione di elaborazione degli orari,
    label=code:PseudoElabora,
    frame=shadowbox,
    rulesepcolor=\color{gray},
    xleftmargin=20pt,
    framexleftmargin=15pt,
    keywordstyle=\color{blue}\bf,
    commentstyle=\color{Green},
    stringstyle=\color{red},
    numbers=left,
    numberstyle=\tiny,
    numbersep=5pt,
    breaklines=true,
    showstringspaces=false,
    basicstyle=\footnotesize,
    emph={fine,funzione,min,max,for,each,end,between, in, and},emphstyle={\color{magenta}}}

\lstinputlisting[language=java]{Fase2/src/Elabora}
%------------------------------------------------
\subsubsection{Analisi della complessità}
Consideriamo:
\begin{itemize}
	\item $k$, il numero di righe inserite nella tabella degli \verb|Orari| aggregati
\end{itemize}
La complessità di questa funzione, considerando le operazioni di accesso al database di costo unitario $O(1)$ (\textit{semplificazione}), può essere considerata pari a
\begin{equation}
	O(k)
\end{equation}
%------------------------------------------------
\subsection{Funzione di eliminazione dei dati grezzi}
Una volta che gli orari sono stati aggregati ed elaborati correttamente, per ottimizzare lo spazio, è richiesta la cancellazione dei dati delle \verb|Timbrature| grezze, importate dal file \verb|XML| prodotto dalla timbratrice.\\

\noindent
Questa operazione, concettualmente molto semplice, viene svolta dallo pseudocodice in Codice \ref{code:Elimina}.

\lstset{
    language=java,
    tabsize=2,
    frame=lines,
    caption=Pseudocodice della funzione di eliminazione delle timbrature grezze,
    label=code:Elimina,
    frame=shadowbox,
    rulesepcolor=\color{gray},
    xleftmargin=20pt,
    framexleftmargin=15pt,
    keywordstyle=\color{blue}\bf,
    commentstyle=\color{Green},
    stringstyle=\color{red},
    numbers=left,
    numberstyle=\tiny,
    numbersep=5pt,
    breaklines=true,
    showstringspaces=false,
    basicstyle=\footnotesize,
    emph={fine,funzione,min,max,for,each,end,between, in, and},emphstyle={\color{magenta}}}

\lstinputlisting[language=java]{Fase2/src/Elimina}
%------------------------------------------------
\subsubsection{Analisi della complessità}
Questa funzione effettua solamente delle operazioni di accesso al database, quindi per le assunzioni fatte in precedenza possiamo considerare la sua complessità pari a
\begin{equation}
	O(1)
\end{equation}
%------------------------------------------------
\subsection{Analisi della complessità generale}
Guardando l'algoritmo di aggregazione ed elaborazione degli orari nella sua completezza, è possibile stabilire la sua complessità come la somma di tutte le complessità delle singole funzioni che lo compongono.\\
In particolare, considerando
\begin{itemize}
	\item $n$, il numero di dipendenti dell'azienda
	\item $m$, il numero di giorni tra \verb|dataMin| e \verb|dataMax|
	\item $k$, il numero di giornate inserite nella tabella degli \verb|Orari| aggregati
	\item Ordinamento eseguito tramite un algoritmo ottimo, con complessità $O(n\cdot \log{n})$.\\
		Questa assunzione vale solamente a livello teorico, in quanto l'ordinamento verrà eseguito tramite una query SQL e, di conseguenza, gestito dal DMBS con algoritmo non-noto.
\end{itemize}
e mantenendo valide tutte le ipotesi semplificative fatte nelle singole analisi, la complessità totale della funzionalità di aggregazione risulta pari a
\begin{equation}
	O(n) + O(n^2 \cdot \log{n}) + O(k) + O(1) = O(n^2 \cdot \log{n})
\end{equation}

