\chapter{Introduzione}
%------------------------------------------------
\section{Scelta della funzione da implementare}
\paragraph{Selezione del caso d'uso da implementare}
Per questa \textit{Fase 2} del processo di sviluppo dell'applicazione si è scelto di implementare una delle due funzionalità chiave del programma, ovvero la funzionalità descritta all'interno del caso d'uso \verb|UC12|: Gestione dei dati provenienti dalla timbratrice.

\paragraph{Formato file di scambio dati}
Come accennato anche all'interno della descrizione del caso d'uso, i dati tra la timbratrice ed il software di gestione delle timbrature vengono scambiati tramite file di tipo \verb|XML|.

\paragraph{Breve descrizione dell'obbiettivo}
Si vuole creare una procedura che permetta la selezione, da parte dell'utente, di un file \verb|XML| prodotto dalla timbratrice e contenente i dati delle timbrature dei dipendenti. L'utente deve inoltre avere la possibilità di aggiungere, modificare o eliminare alcuni dei dati e, successivamente, confermare le modifiche effettuando l'aggregazione dei dati.

\paragraph{Dati di input}
Gli unici dati di input sono contenuti all'interno del file \verb|XML|:
\begin{itemize}
	\item Data timbratura
	\item Tipo di timbratura (\verb|IN| o \verb|OUT|)
	\item Identificativo del badge
	\item Orario di timbratura
\end{itemize}
%------------------------------------------------
\section{Passi principali della funzionalità}
I passi principali dei quali si compone la funzionalità da implementare saranno:
\begin{itemize}
	\item Scelta del file \verb|XML|
	\item Conversione del file \verb|XML| in strutture dati (\textit{parsing})
	\item Visualizzazione dei dati parsati, con relativa possibilità di:
		\begin{itemize}
			\item Cancellazione dato
			\item Modifica dato
		\end{itemize}
	\item Salvataggio dei dati modificati manualmente
	\item Aggregazione degli orari e salvataggio definitivo
\end{itemize}