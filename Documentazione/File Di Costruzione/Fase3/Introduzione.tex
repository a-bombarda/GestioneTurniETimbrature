\chapter{Introduzione}
%------------------------------------------------
\section{Scelta della funzione da implementare}
\paragraph{Selezione del caso d'uso da implementare}
Per questa \textit{Fase 3} del processo di sviluppo dell'applicazione si è scelto di implementare la seconda delle due funzionalità chiave del programma, ovvero la funzionalità descritta all'interno del caso d'uso \verb|UC8|: Generazione turnazione settimanale.

\paragraph{Breve descrizione dell'obbiettivo}
Si vuole creare una procedura che, considerando tutti i dipendenti dell'azienda committente, sia in grado di allocare i turni programmati per la settimana, rispettando il numero massimo di ore che possono essere lavorate da ciascun dipendente ogni settimana e rispettando il fabbisogno per ciascuna attività.

\paragraph{Dati di input}
I dati di input necessari per il funzionamento della funzionalità analizzata in questa fase sono:
\begin{itemize}
	\item Lista dei dipendenti dell'azienda committente.
		\begin{itemize}
			\item Numero massimo di ore settimanali per ciascun dipendente.
		\end{itemize}
	\item Lista delle attività da svolgere.
	\item Lista dei turni.
	\item Fabbisogno di dipendenti, per ciascun turno, in ciascuna data.
\end{itemize}

\paragraph{Dati di output}
L'output prodotto dall'algoritmo è un piano di allocazione dei turni, che rispetta i vincoli che sono stati posti.
%------------------------------------------------
\section{Aggiornamento dei dati nel database}
Anche per il caso d'uso \verb|UC8| sono necessari dei dati, all'interno del database, aggiuntivi rispetto a quelli inseriti nella \textit{Fase 2}.\\
Nella versione finale del software, questi dati saranno inseribili da UI da parte dell'utente utilizzatore. A questo punto del processo di sviluppo, però, le funzionalità di inserimento dati non sono ancora state implementate, quindi si è scelto di realizzare uno script \verb|SQL| per l'inserimento di dati di prova (Codice \ref{code:InsertSQL}). \\

\lstset{
    language=XML,
    tabsize=2,
    frame=lines,
    caption=Query di aggiornamento dei dati nel database,
    label=code:InsertSQL,
    frame=shadowbox,
    rulesepcolor=\color{gray},
    xleftmargin=20pt,
    framexleftmargin=15pt,
    morekeywords={xmlns,version,type,encoding},
    keywordstyle=\color{blue}\bf,
    stringstyle=\color{red},
    numbers=left,
    numberstyle=\tiny,
    numbersep=5pt,
    breaklines=true,
    showstringspaces=false,
    basicstyle=\footnotesize,
    emph={records,record,badgeID,time,xs:schema, xs:element, xs:complexType, xs:attribute, xs:sequence},emphstyle={\color{magenta}}}

\lstinputlisting[language=SQL]{Fase3/src/AggiornamentoDB.sql}